% Options for packages loaded elsewhere
\PassOptionsToPackage{unicode}{hyperref}
\PassOptionsToPackage{hyphens}{url}
\documentclass[
]{article}
\usepackage{xcolor}
\usepackage[margin=1in]{geometry}
\usepackage{amsmath,amssymb}
\setcounter{secnumdepth}{-\maxdimen} % remove section numbering
\usepackage{iftex}
\ifPDFTeX
  \usepackage[T1]{fontenc}
  \usepackage[utf8]{inputenc}
  \usepackage{textcomp} % provide euro and other symbols
\else % if luatex or xetex
  \usepackage{unicode-math} % this also loads fontspec
  \defaultfontfeatures{Scale=MatchLowercase}
  \defaultfontfeatures[\rmfamily]{Ligatures=TeX,Scale=1}
\fi
\usepackage{lmodern}
\ifPDFTeX\else
  % xetex/luatex font selection
    \setmainfont[]{DejaVu Serif}
    \setmonofont[]{DejaVu Sans Mono}
\fi
% Use upquote if available, for straight quotes in verbatim environments
\IfFileExists{upquote.sty}{\usepackage{upquote}}{}
\IfFileExists{microtype.sty}{% use microtype if available
  \usepackage[]{microtype}
  \UseMicrotypeSet[protrusion]{basicmath} % disable protrusion for tt fonts
}{}
\makeatletter
\@ifundefined{KOMAClassName}{% if non-KOMA class
  \IfFileExists{parskip.sty}{%
    \usepackage{parskip}
  }{% else
    \setlength{\parindent}{0pt}
    \setlength{\parskip}{6pt plus 2pt minus 1pt}}
}{% if KOMA class
  \KOMAoptions{parskip=half}}
\makeatother
\ifLuaTeX
\usepackage[bidi=basic]{babel}
\else
\usepackage[bidi=default]{babel}
\fi
\babelprovide[main,import]{spanish}
\ifPDFTeX
\else
\babelfont{rm}[]{DejaVu Serif}
\fi
% get rid of language-specific shorthands (see #6817):
\let\LanguageShortHands\languageshorthands
\def\languageshorthands#1{}
\setlength{\emergencystretch}{3em} % prevent overfull lines
\providecommand{\tightlist}{%
  \setlength{\itemsep}{0pt}\setlength{\parskip}{0pt}}
\usepackage{bookmark}
\IfFileExists{xurl.sty}{\usepackage{xurl}}{} % add URL line breaks if available
\urlstyle{same}
\hypersetup{
  pdftitle={Bienvenida --- Matemáticas Discretas para Ciencia de Datos},
  pdflang={es},
  hidelinks,
  pdfcreator={LaTeX via pandoc}}

\title{Bienvenida --- Matemáticas Discretas para Ciencia de Datos}
\author{}
\date{}


        \usepackage{fontspec}
        \usepackage{unicode-math}
        \setmainfont{DejaVu Serif}
        \setmonofont{DejaVu Sans Mono}
        \usepackage{amsmath}
        \usepackage{amssymb}
        \usepackage{graphicx}
        \usepackage{hyperref}
        \usepackage[spanish]{babel}
        
\begin{document}
\maketitle

\phantomsection\label{main-content}
{}

\emph{}

\begin{itemize}
\tightlist
\item
  \href{_sources/intro.md}{{ \emph{} } {.md}}
\item
  { \emph{} } {.pdf}
\end{itemize}

{ \emph{} }

{}

\phantomsection\label{jb-print-docs-body}
\section{Bienvenida}\label{bienvenida}

\phantomsection\label{print-main-content}
\phantomsection\label{jb-print-toc}
\subsection{Contenido}\label{contenido}

\begin{itemize}
\tightlist
\item
  \hyperref[preambulo]{Preámbulo}
\item
  \hyperref[pre-requisitos]{Pre-requisitos}
\item
  \hyperref[temario]{Temario}
\end{itemize}

\phantomsection\label{searchbox}

\phantomsection\label{bienvenida}
\section{\texorpdfstring{Bienvenida\hyperref[bienvenida]{\#}}{Bienvenida\#}}\label{bienvenida-1}

¡Bienvenido al libro online de Matemáticas Discretas para Ciencia de
Datos!

\phantomsection\label{preambulo}
\subsection{\texorpdfstring{Preámbulo\hyperref[preambulo]{\#}}{Preámbulo\#}}\label{preuxe3mbulo}

Dos de las habilidades complementarias de mayor utilidad en una carrera
de Ciencia de Datos son el uso de teoría de gráficas y el diseño de
algoritmos. Sin embargo, usualmente se requiere de mucho tiempo para
entender con profundidad los fundamentos matemáticos y de ciencias de
la computación detrás de estos temas. En ocasiones, entender todos los
detalles requiere de tomar varios cursos a nivel universitario. Si bien
esto llevaría a un entendimiento a detalle de la teoría, usualmente no
hay el tiempo necesario para aprender todo esto dentro de los ambiciosos
programas de estudio en ciencias de datos.

El objetivo de este libro es dar una muy buena introducción a estos
temas. El tratamiento no tiene la profundidad de tomar varios cursos
completos. Sin embargo, el material sí es suficiente para desarrollar un
fuerte nivel de abstracción y para cubrir la mayoría de las necesidades
prácticas de un científico de datos.

Se cubre el material desde dos perspectivas. Por un lado, se desarrolla
teoría matemática y computacional de manera formal, dando definiciones,
ejemplos, proposiciones, teoremas y demostraciones cada que sea
pertinente. Por otro lado, se desarrolla el aspecto práctico mediante
descripciones de algoritmos, pseudocódigos y numerosas celdas para
ejecutar. El lenguaje de programación elegido es Python y se usan
varias librerías abiertas.

Al final de cada tema se presenta una sección de Tarea Moral, cuyo
objetivo es continuar practicando de manera autodidacta el contenido
cubierto. Además, al final de cada capítulo se presenta una lista de
ejercicios teóricos y de implementación.

\subsection{\texorpdfstring{Pre-requisitos\hyperref[pre-requisitos]{\#}}{Pre-requisitos\#}}\label{pre-requisitos}

En términos de matemáticas, supondremos que el lector tiene una
formación buena correspondiente a los primeros dos semestres de una
licenciatura en matemáticas o algo afín. Específicamente, supondremos
cierta familiaridad con los temas de inducción, sucesiones, conteo y
cálculo. Supondremos también que el lector está acostumbrado a seguir
una estructura de definiciones, proposiciones, lemas, teoremas,
corolarios. En términos de computación teórica no supondremos
pre-requisitos.

La mayor parte del texto puede leerse sólo de manera teórica. Sin
embargo, para entender con profundidad los temas y llevarlos a la
práctica, recomendamos que el lector estudie y juegue con todo el
código en Python que se propone a lo largo del libro. Para ello,
supondremos un manejo básico de Python: tipos de datos, asignaciones,
comparaciones, ciclos, condicionales, etc.

\subsection{\texorpdfstring{Temario\hyperref[temario]{\#}}{Temario\#}}\label{temario}

Nota

Este es un libro en elaboración. En el siguiente temario puedes
encontrar indicados los capítulos con más progreso.

Parte 1: Fundamentos combinatorios

Buscar un patrón

Principio de las casillas

Principio de doble conteo y coeficientes binomiales

Principio de inducción

Principio de recursión y recursiones lineales

Principio extremo

Parte 2: Teoría de gráficas

Gráficas, grados y lema de Euler

Caminos, conexidad y distancia

Caminos cerrados, circuitos y ciclos

Recorrer toda una gráfica

Ã?rboles y bosques

Gráficas bipartitas

Emparejamientos y teorema de Hall

Conjuntos independientes y coloraciones

Coloración de aristas y teorema de Ramsey

Subgráficas completas y teorema de Turán

Parte 3: Diseño y análisis de algoritmos

Problemas y algoritmos

Algoritmos incorrectos

Algoritmos correctos

Modelo RAM y pensamiento asintótico

Notación O grande y similares

Propiedades de la notación O grande

Ejemplos de análisis de eficiencia

El problema de ordenar

Ordenar cuadráticamente

Estructuras de datos vs tipos de datos abstractos

Diccionarios y árboles de búsqueda balanceados

Ordenar eficientemente

Aplicaciones de ordenar

Parte 4: Heurísticas de creación de algoritmos

Tipos de problemas algorítmicos

Espacio de estados

Exploración exhaustiva

Recortes al espacio de estados

Algoritmos voraces

Divide y conquista

Recursión y teorema maestro

Backtrack en búsquedas combinatorias

Más ejemplos de backtrack

Programación dinámica

Métodos probabilistas

Parte 5: Algoritmos en teoría de gráficas

Implementaciones de gráficas y variantes

Uso básico de NetworkX

Búsqueda por anchura

Aplicaciones de búsqueda por anchura

Búsqueda por profundidad

Aplicaciones de búsqueda por profundidad

�rboles de peso mínimo: algoritmos de Prim y Kruskal

Caminos de peso mínimo: algoritmos de Dijkstra y Floyd-Warshall

Redes, flujos y flujos máximos

Algoritmos de Ford-Fulkerson y Edmonds-Karp

Reducciones algorítmicas y clases P, NP y NP-completo

Problemas de gráficas NP-completos

\href{P1/IntroCombinatoria.html}{}

siguiente

Fundamentos de combinatoria

\emph{}

\emph{} Contenido

\begin{itemize}
\tightlist
\item
  \hyperref[preambulo]{Preámbulo}
\item
  \hyperref[pre-requisitos]{Pre-requisitos}
\item
  \hyperref[temario]{Temario}
\end{itemize}

Por Leonardo Ignacio Martínez Sandoval

© Copyright 2022.\\

\end{document}
