% Options for packages loaded elsewhere
\PassOptionsToPackage{unicode}{hyperref}
\PassOptionsToPackage{hyphens}{url}
\documentclass[
]{article}
\usepackage{xcolor}
\usepackage[margin=1in]{geometry}
\usepackage{amsmath,amssymb}
\setcounter{secnumdepth}{-\maxdimen} % remove section numbering
\usepackage{iftex}
\ifPDFTeX
  \usepackage[T1]{fontenc}
  \usepackage[utf8]{inputenc}
  \usepackage{textcomp} % provide euro and other symbols
\else % if luatex or xetex
  \usepackage{unicode-math} % this also loads fontspec
  \defaultfontfeatures{Scale=MatchLowercase}
  \defaultfontfeatures[\rmfamily]{Ligatures=TeX,Scale=1}
\fi
\usepackage{lmodern}
\ifPDFTeX\else
  % xetex/luatex font selection
    \setmainfont[]{DejaVu Serif}
    \setmonofont[]{DejaVu Sans Mono}
\fi
% Use upquote if available, for straight quotes in verbatim environments
\IfFileExists{upquote.sty}{\usepackage{upquote}}{}
\IfFileExists{microtype.sty}{% use microtype if available
  \usepackage[]{microtype}
  \UseMicrotypeSet[protrusion]{basicmath} % disable protrusion for tt fonts
}{}
\makeatletter
\@ifundefined{KOMAClassName}{% if non-KOMA class
  \IfFileExists{parskip.sty}{%
    \usepackage{parskip}
  }{% else
    \setlength{\parindent}{0pt}
    \setlength{\parskip}{6pt plus 2pt minus 1pt}}
}{% if KOMA class
  \KOMAoptions{parskip=half}}
\makeatother
\usepackage{longtable,booktabs,array}
\usepackage{calc} % for calculating minipage widths
% Correct order of tables after \paragraph or \subparagraph
\usepackage{etoolbox}
\makeatletter
\patchcmd\longtable{\par}{\if@noskipsec\mbox{}\fi\par}{}{}
\makeatother
% Allow footnotes in longtable head/foot
\IfFileExists{footnotehyper.sty}{\usepackage{footnotehyper}}{\usepackage{footnote}}
\makesavenoteenv{longtable}
\ifLuaTeX
\usepackage[bidi=basic]{babel}
\else
\usepackage[bidi=default]{babel}
\fi
\babelprovide[main,import]{spanish}
\ifPDFTeX
\else
\babelfont{rm}[]{DejaVu Serif}
\fi
% get rid of language-specific shorthands (see #6817):
\let\LanguageShortHands\languageshorthands
\def\languageshorthands#1{}
\setlength{\emergencystretch}{3em} % prevent overfull lines
\providecommand{\tightlist}{%
  \setlength{\itemsep}{0pt}\setlength{\parskip}{0pt}}
\usepackage{bookmark}
\IfFileExists{xurl.sty}{\usepackage{xurl}}{} % add URL line breaks if available
\urlstyle{same}
\hypersetup{
  pdftitle={Principio de doble conteo y coeficientes binomiales --- Matemáticas Discretas para Ciencia de Datos},
  pdflang={es},
  hidelinks,
  pdfcreator={LaTeX via pandoc}}

\title{Principio de doble conteo y coeficientes binomiales ---
Matemáticas Discretas para Ciencia de Datos}
\author{}
\date{}


        \usepackage{fontspec}
        \usepackage{unicode-math}
        \setmainfont{DejaVu Serif}
        \setmonofont{DejaVu Sans Mono}
        \usepackage{amsmath}
        \usepackage{amssymb}
        \usepackage{graphicx}
        \usepackage{hyperref}
        \usepackage[spanish]{babel}
        
\begin{document}
\maketitle

\phantomsection\label{main-content}
{}

\emph{}

\begin{itemize}
\tightlist
\item
  \href{../_sources/P1/DobleConteo.ipynb}{{ \emph{} } {.ipynb}}
\item
  { \emph{} } {.pdf}
\end{itemize}

{ \emph{} }

{}

\phantomsection\label{jb-print-docs-body}
\section{Principio de doble conteo y coeficientes
binomiales}\label{principio-de-doble-conteo-y-coeficientes-binomiales}

\phantomsection\label{print-main-content}
\phantomsection\label{jb-print-toc}
\subsection{Contenido}\label{contenido}

\begin{itemize}
\tightlist
\item
  \hyperref[introduccion]{Introducción}
\item
  \hyperref[principio-de-doble-conteo]{Principio de doble conteo}
\item
  \hyperref[coeficientes-binomiales-y-su-formula]{Coeficientes
  binomiales y su fórmula}
\item
  \hyperref[la-formula-de-pascal]{La fórmula de Pascal}
\item
  \hyperref[exploracion-computacional-del-triangulo-de-pascal]{Exploración
  computacional del triángulo de Pascal}
\item
  \hyperref[tarea-moral]{Tarea moral}
\end{itemize}

\phantomsection\label{searchbox}

\phantomsection\label{principio-de-doble-conteo-y-coeficientes-binomiales}
\section{\texorpdfstring{Principio de doble conteo y coeficientes
binomiales\hyperref[principio-de-doble-conteo-y-coeficientes-binomiales]{\#}}{Principio de doble conteo y coeficientes binomiales\#}}\label{principio-de-doble-conteo-y-coeficientes-binomiales-1}

\phantomsection\label{introduccion}
\subsection{\texorpdfstring{Introducción\hyperref[introduccion]{\#}}{Introducción\#}}\label{introducciuxe3uxb3n}

En esta entrada supondremos que estás familiarizado con las nociones
básicas de conteo combinatorio: la enumeración, la regla de la suma,
la regla del producto y las permutaciones. En caso de no ser así, te
recomendamos revisar el siguiente material:

\begin{itemize}
\item
  Los videos de Combinatoria en la página
  \url{https://blog.nekomath.com/videos/}
\item
  El artículo «Estrategias básicas de conteo» en la revista Tzaloa
  2011-2:\url{https://www.ommenlinea.org/wp-content/uploads/2011/06/Tzaloa-2-2011.pdf}
\end{itemize}

Lo que haremos ahora es introducir un principio nuevo: el de doble
conteo. La idea es muy sencilla: dos formas distintas y válidas de
contar una familia de objetos deben de dar cantidades iguales. Al contar
de manera creativa, esto se puede utilizar para la resolución de muchos
problemas. En particular, veremos cómo estas ideas se aplican al
desarrollo de la teoría de coeficientes binomiales.

\subsection{\texorpdfstring{Principio de doble
conteo\hyperref[principio-de-doble-conteo]{\#}}{Principio de doble conteo\#}}\label{principio-de-doble-conteo}

El principio de doble conteo dice lo siguiente.

\textbf{Proposición.} Supongamos que tenemos un conjunto
{\textbackslash(X\textbackslash)} de objetos. Supongamos que dos
estrategias de conteo distintas, pero válidas, nos dicen que
{\textbackslash(X\textbackslash)} tiene por un lado
{\textbackslash(A\textbackslash)} elementos y por otro lado
{\textbackslash(B\textbackslash)} elementos. Entonces,
{\textbackslash(A=B\textbackslash)}.

Probemos con este principio un resultado clásico.

\textbf{Problema.} Muestra que

\textbackslash{[}1+2+3+\textbackslash ldots+n=\textbackslash frac\{n(n+1)\}\{2\}.\textbackslash{]}

\emph{Solución.} Realicemos una demostración por doble conteo. Para
ello, introducimos un problema auxiliar. Contaremos cuántas parejas
{\textbackslash((a,b)\textbackslash)} existen para las cuales
{\textbackslash(a\textbackslash)} es distinto de
{\textbackslash(b\textbackslash)} y {\textbackslash(a,b\textbackslash)}
son números de {\textbackslash(1\textbackslash)} a
{\textbackslash(n+1\textbackslash)}.

Una forma de contar es la siguiente. La primera entrada tiene
{\textbackslash(n+1\textbackslash)} posibilidades. Una vez fija, la
segunda entrada tiene {\textbackslash(n\textbackslash)} posibilidades.
Así, por el principio del producto una posible respuesta al problema de
conteo es {\textbackslash(n(n+1)\textbackslash)}.

Otra forma de contar es la siguiente. Supongamos de momento que
{\textbackslash(a\textgreater b\textbackslash)}.

\begin{itemize}
\item
  Si {\textbackslash(a=1\textbackslash)}, entonces
  {\textbackslash(b\textbackslash)} no tiene posiblidades.
\item
  Si {\textbackslash(a=2\textbackslash)}, entonces
  {\textbackslash(b\textbackslash)} tiene una posibilidad.
\item
  Si {\textbackslash(a=3\textbackslash)}, entonces
  {\textbackslash(b\textbackslash)} tiene dos posibilidades.
\item
  {\textbackslash(\textbackslash ldots\textbackslash)}
\item
  Si {\textbackslash(a=n+1\textbackslash)}, entonces
  {\textbackslash(b\textbackslash)} tiene
  {\textbackslash(n+1\textbackslash)} posibilidades.
\end{itemize}

Así, hay {\textbackslash(1+2+\textbackslash ldots+n\textbackslash)}
posiblidades. Pero faltan los casos con
{\textbackslash(b\textgreater a\textbackslash)}, que por simetría son
los mismos. Así, tenemos que:

\textbackslash{[}2(1+2+\textbackslash ldots+n) = \textbackslash text\{
número de parejas (a,b) \} =n(n+1).\textbackslash{]}

Dividiendo entre {\textbackslash(2\textbackslash)} obtenemos el
resultado deseado.
{{\textbackslash(\textbackslash square\textbackslash)}}

\phantomsection\label{coeficientes-binomiales-y-su-formula}
\subsection{\texorpdfstring{Coeficientes binomiales y su
fórmula\hyperref[coeficientes-binomiales-y-su-formula]{\#}}{Coeficientes binomiales y su fórmula\#}}\label{coeficientes-binomiales-y-su-fuxe3uxb3rmula}

Quizás hayas encontrado con anterioridad a los coeficientes binomiales.
Para un entero {\textbackslash(n\textbackslash geq 0\textbackslash)} y
un entero {\textbackslash(k\textbackslash)} con
{\textbackslash(0\textbackslash leq k \textbackslash leq
n\textbackslash)}, se puede definir el símbolo
{\textbackslash(\textbackslash binom\{n\}\{k\}\textbackslash)} de varias
maneras. Nosotros eligiremos definir al coeficiente binomial
{\textbackslash(\textbackslash binom\{n\}\{k\}\textbackslash)} a partir
de su propiedad combinatoria fundamental.

\textbf{Definición.} El coeficiente binomial
{\textbackslash(\textbackslash binom\{n\}\{k\}\textbackslash)} es la
cantidad de subconjuntos de tamaño {\textbackslash(k\textbackslash)}
que tiene un conjunto de tamaño {\textbackslash(n\textbackslash)}.

Así, estrictamente hablando, si quisiéramos calcular cuánto vale el
coeficiente binomial
{\textbackslash(\textbackslash binom\{4\}\{2\}\textbackslash)},
tendríamos que tomar un conjunto con cuatro elementos, por ejemplo
{\textbackslash(\textbackslash\{A,B,C,D\textbackslash\}\textbackslash)},
enlistar todos sus subconjuntos con dos elementos:

\textbackslash{[}\textbackslash\{A,B\textbackslash\},\textbackslash\{A,C\textbackslash\},\textbackslash\{A,D\textbackslash\},\textbackslash\{B,C\textbackslash\},\textbackslash\{B,D\textbackslash\},\textbackslash\{C,D\textbackslash\},\textbackslash{]}

y contarlos. Como son {\textbackslash(6\textbackslash)}, entonces
{\textbackslash(\textbackslash binom\{4\}\{2\}=6\textbackslash)}.

Esto no es muy práctico para valores grandes de
{\textbackslash(n\textbackslash)} y {\textbackslash(k\textbackslash)},
pues tendríamos que enlistar muchos subconjuntos. Así, es importante
encontrar una mejor fórmula para
{\textbackslash(\textbackslash binom\{n\}\{k\}\textbackslash)}. Lo
hacemos como sigue:

\textbf{Proposición.} Se tiene que

\textbackslash{[}\textbackslash binom\{n\}\{k\}=\textbackslash frac\{n!\}\{k!(n-k)!\}.\textbackslash{]}

\emph{Demostración.} Tomemos pelotas numeradas del
{\textbackslash(1\textbackslash)} al {\textbackslash(n\textbackslash)}.
¿Cuántas formas hay de ordenarlas en una línea? Por un lado, hay
{\textbackslash(n!\textbackslash)} formas:
{\textbackslash(n\textbackslash)} opciones para el primer lugar,
{\textbackslash(n-1\textbackslash)} opciones para el segundo, etc. Estos
números se multiplican y obtenemos el valor
{\textbackslash(n\textbackslash cdot (n-1) \textbackslash cdot
(n-2)\textbackslash cdot \textbackslash ldots \textbackslash cdot 1 =
n!\textbackslash)}.

Hay otra forma de saber de cuántas formas podemos ordenar las pelotas.
Primero podemos decidir cuáles serán las primeras
{\textbackslash(k\textbackslash)} pelotas. Esto se puede hacer de
{\textbackslash(\textbackslash binom\{n\}\{k\}\textbackslash)} formas,
pues es elegir un subconjunto {\textbackslash(K\textbackslash)} de
tamaño {\textbackslash(k\textbackslash)}. Luego, hay que elegir una
forma de acomodar estas {\textbackslash(k\textbackslash)} pelotas, que
se puede hacer de {\textbackslash(k!\textbackslash)} formas (por el
arugmento de arriba). Finalmente, quedan
{\textbackslash(n-k\textbackslash)} pelotas que debemos decir cómo
acomodar. Esto se puede hacer de {\textbackslash((n-k)!\textbackslash)}
formas. Así, hay
{\textbackslash(\textbackslash binom\{n\}\{k\}k!(n-k)!\textbackslash)}
formas de acomodar las pelotas.

Como estamos contando exactamente lo mismo, debemos tener que

\textbackslash{[}n!=\textbackslash binom\{n\}\{k\}k!(n-k)!,\textbackslash{]}

de donde se obtiene la fórmula que buscamos despejando
{\textbackslash(\textbackslash binom\{n\}\{k\}\textbackslash)}.
{{\textbackslash(\textbackslash square\textbackslash)}}

\phantomsection\label{la-formula-de-pascal}
\subsection{\texorpdfstring{La fórmula de
Pascal\hyperref[la-formula-de-pascal]{\#}}{La fórmula de Pascal\#}}\label{la-fuxe3uxb3rmula-de-pascal}

Los coeficientes binomiales con {\textbackslash(k=0\textbackslash)} o
{\textbackslash(k=n\textbackslash)} son muy fáciles: simplemente son
iguales a {\textbackslash(1\textbackslash)}. Para calcular el resto de
los coeficientes binomiales podemos proceder «poco a poco». Cada uno
de ellos es una suma de dos «más pequeños». Esto se formaliza
mediante el siguiente resultado.

\textbf{Proposición.} Para un entero
{\textbackslash(n\textbackslash geq 0\textbackslash)} y un entero
{\textbackslash(k\textbackslash)} con
{\textbackslash(0\textbackslash leq k \textbackslash leq
n-1\textbackslash)} se tiene que

\textbackslash{[}\textbackslash binom\{n+1\}\{k+1\}=\textbackslash binom\{n\}\{k\}+\textbackslash binom\{n\}\{k+1\}.\textbackslash{]}

A esta se le conoce como la \textbf{identidad de Pascal} para los
coeficientes binomiales. Observa que nos está diciendo que los
coeficientes binomiales justo cumplen la regla del triángulo de Pascal:
cada uno es o bien igual a uno (en los extremos) o bien la suma de dos
anteriores con un valor menor que {\textbackslash(n\textbackslash)} (es
decir «en el renglón anterior»). Una vez que hayamos demostrado esta
fórmula, tendremos la garantía de que el triángulo de Pascal se ve
así:

\begin{longtable}[]{@{}llllllllllll@{}}
\toprule\noalign{}
& & & & & & & & & & & \\
\midrule\noalign{}
\endhead
\bottomrule\noalign{}
\endlastfoot
& & & & & {\textbackslash(\textbackslash binom\{0\}\{0\}\textbackslash)}
& & & & & & \\
& & & & {\textbackslash(\textbackslash binom\{1\}\{0\}\textbackslash)} &
& {\textbackslash(\textbackslash binom\{1\}\{1\}\textbackslash)} & & & &
& \\
& & & {\textbackslash(\textbackslash binom\{2\}\{0\}\textbackslash)} & &
{\textbackslash(\textbackslash binom\{2\}\{1\}\textbackslash)} & &
{\textbackslash(\textbackslash binom\{2\}\{2\}\textbackslash)} & & &
& \\
& & {\textbackslash(\textbackslash binom\{3\}\{0\}\textbackslash)} & &
{\textbackslash(\textbackslash binom\{3\}\{1\}\textbackslash)} & &
{\textbackslash(\textbackslash binom\{3\}\{2\}\textbackslash)} & &
{\textbackslash(\textbackslash binom\{3\}\{3\}\textbackslash)} & & & \\
& {\textbackslash(\textbackslash binom\{4\}\{0\}\textbackslash)} & &
{\textbackslash(\textbackslash binom\{4\}\{1\}\textbackslash)} & &
{\textbackslash(\textbackslash binom\{4\}\{2\}\textbackslash)} & &
{\textbackslash(\textbackslash binom\{4\}\{3\}\textbackslash)} & &
{\textbackslash(\textbackslash binom\{4\}\{4\}\textbackslash)} & & \\
{\textbackslash(\textbackslash binom\{5\}\{0\}\textbackslash)} & &
{\textbackslash(\textbackslash binom\{5\}\{1\}\textbackslash)} & &
{\textbackslash(\textbackslash binom\{5\}\{2\}\textbackslash)} & &
{\textbackslash(\textbackslash binom\{5\}\{3\}\textbackslash)} & &
{\textbackslash(\textbackslash binom\{5\}\{4\}\textbackslash)} & &
{\textbackslash(\textbackslash binom\{5\}\{5\}\textbackslash)} & \\
\end{longtable}

Veamos dos maneras de demostrar la identidad de Pascal. Una es
\emph{combinatoria} y la otra es \emph{algebraica}.

\emph{Demostración.} La demostración combinatoria es por doble conteo.
Podemos hacernos la siguiente pregunta: ¿cuántos subconjuntos de
tamaño {\textbackslash(k+1\textbackslash)} tiene el conjunto
{\textbackslash(\textbackslash\{1,2,\textbackslash ldots,n+1\textbackslash\}\textbackslash)}?
Por un lado, por la definición de coeficientes binomiales este número
es {\textbackslash(\textbackslash binom\{n+1\}\{k+1\}\textbackslash)}.
Por otro lado, podemos ver cuántos tienen a
{\textbackslash(n+1\textbackslash)} y cuántos no. Los que tienen a
{\textbackslash(n+1\textbackslash)} consisten de ese elemento y de otros
{\textbackslash(k\textbackslash)} elementos elegidos en
{\textbackslash(\textbackslash\{1,2,\textbackslash ldots,n\textbackslash\}\textbackslash)},
así que son
{\textbackslash(\textbackslash binom\{n\}\{k\}\textbackslash)}. Los que
no tienen a {\textbackslash(n+1\textbackslash)} consisten de elegir
{\textbackslash(k+1\textbackslash)} elementos en
{\textbackslash(\textbackslash\{1,2\textbackslash ldots,n\textbackslash\}\textbackslash)},
de modo que son
{\textbackslash(\textbackslash binom\{n\}\{k+1\}\textbackslash)}. Así,
como estamos contando la misma cosa, tenemos que

\textbackslash{[}\textbackslash binom\{n+1\}\{k+1\}=\textbackslash binom\{n\}\{k\}+\textbackslash binom\{n\}\{k+1\}.\textbackslash{]}

{{\textbackslash(\textbackslash square\textbackslash)}}

Esto es una demostración perfectamente válida y podríamos concluir
aquí. Pero demos ahora una demostración algebraica. Para ella, usamos
la fórmula que ya conocemos.

\emph{Demostación.} Procedemos como sigue:

\textbackslash{[}\textbackslash begin\{align*\}
\textbackslash binom\{n\}\{k\}+\textbackslash binom\{n\}\{k+1\}\&=\textbackslash frac\{n!\}\{k!(n-k)!\}+\textbackslash frac\{n!\}\{(k+1)!(n-k-1)!\}\textbackslash\textbackslash{}
\&=\textbackslash frac\{n!\}\{k!(n-k-1)!\}\textbackslash left(\textbackslash frac\{1\}\{n-k\}+\textbackslash frac\{1\}\{k+1\}\textbackslash right)\textbackslash\textbackslash{}
\&=\textbackslash frac\{n!\}\{k!(n-k-1)!\}\textbackslash cdot
\textbackslash frac\{n+1\}\{(n-k)(k+1)\}\textbackslash\textbackslash{}
\&=\textbackslash frac\{(n+1)!\}\{(k+1)!(n-k)!\}\textbackslash\textbackslash{}
\&=\textbackslash binom\{n+1\}\{k+1\}.
\textbackslash end\{align*\}\textbackslash{]}

En la segunda igualdad factorizamos lo más que podemos. Luego hacemos
la operación con fracciones, reagrupamos y volvemos a usar la fórmula,
pero ahora para
{\textbackslash(\textbackslash binom\{n+1\}\{k+1\}\textbackslash)}.
{{\textbackslash(\textbackslash square\textbackslash)}}

\phantomsection\label{exploracion-computacional-del-triangulo-de-pascal}
\subsection{\texorpdfstring{Exploración computacional del triángulo de
Pascal\hyperref[exploracion-computacional-del-triangulo-de-pascal]{\#}}{Exploración computacional del triángulo de Pascal\#}}\label{exploraciuxe3uxb3n-computacional-del-triuxe3ngulo-de-pascal}

Una de las aplicaciones de la fórmula de Pascal es que nos permite
escribir código para calcular los coeficientes binomiales, y por lo
tanto para mostrar el triángulo de Pascal. En la siguiente celda
podrás encontrar una forma de hacer esto. La función
{\texttt{t\_pascal}} muestra los primeros
{\textbackslash(n\textbackslash)} renglones del triángulo de Pascal. El
primero y segundo renglón son fáciles de dar de manera explícita. Los
siguientes renglones son calculados usando la fórmula de Pascal. No
entraremos mucho más en los detalles de lo que está sucediendo, pues
los discutiremos más adelante cuando hablamos de programación
dinámica.

\begin{verbatim}
def t_pascal(n):
    renglon=[1]
    for j in range(n):
        nuevo_renglon=[]
        for k in range(j+1):
            if k==0 or k==j:
                nuevo_renglon.append(1)
            else:
                nuevo_renglon.append(renglon[k-1]+renglon[k])
        print(nuevo_renglon)
        renglon=nuevo_renglon

t_pascal(10)
\end{verbatim}

\begin{verbatim}
[1]
[1, 1]
[1, 2, 1]
[1, 3, 3, 1]
[1, 4, 6, 4, 1]
[1, 5, 10, 10, 5, 1]
[1, 6, 15, 20, 15, 6, 1]
[1, 7, 21, 35, 35, 21, 7, 1]
[1, 8, 28, 56, 70, 56, 28, 8, 1]
[1, 9, 36, 84, 126, 126, 84, 36, 9, 1]
\end{verbatim}

\subsection{\texorpdfstring{Tarea
moral\hyperref[tarea-moral]{\#}}{Tarea moral\#}}\label{tarea-moral}

Los siguientes problemas te ayudarán a practicar lo visto en esta
entrada. Para resolverlos, necesitarás usar herramientas matemáticas,
computacionales o ambas.

\begin{enumerate}
\item
  Da una demostración por doble conteo para la siguiente igualdad:

  \textbackslash{[}1\textbackslash cdot 1! + 2\textbackslash cdot 2! +
  3\textbackslash cdot 3! + \textbackslash ldots + n\textbackslash cdot
  n! = (n+1)! - 1.\textbackslash{]}
\item
  Realiza una exploración computacional para conjeturar cuáles son los
  renglones del triángulo de Pascal en donde todos los números son
  impares.
\item
  La siguiente igualdad siempre se cumple:

  \textbackslash{[}\textbackslash binom\{n\}\{1\}+2\textbackslash binom\{n\}\{2\}+3\textbackslash binom\{n\}\{3\}+\textbackslash ldots+n\textbackslash binom\{n\}\{n\}=n2\^{}\{n-1\}.\textbackslash{]}

  Verifica esto de manera computacional hasta
  {\textbackslash(n=10\textbackslash)}. Luego, da una demostración
  combinatoria (por doble conteo) y una algebraica (por inducción).
\item
  Lee el artículo «Contando de dos formas distintas» en la revista
  Tzaloa 2011-3 para conocer más acerca del principio de doble conteo:
  \url{https://www.ommenlinea.org/wp-content/uploads/2011/09/Tzaloa-3-2011.pdf}
\item
  La siguente expresión se puede simplificar a un único coeficiente
  binomial:

  \textbackslash{[}\textbackslash binom\{n\}\{0\}\^{}2+\textbackslash binom\{n\}\{1\}\^{}2+\textbackslash ldots+\textbackslash binom\{n\}\{n\}\^{}2.\textbackslash{]}

  ¿A cuál? Para determinar esto, realiza una exploración
  computacional que calcule la expresión en los primeros casos. Luego,
  compara el resultado con otros coeficientes binomiales. Haz una
  conjetura. Finalmente, realiza una demostración por doble conteo de
  la identidad que propones.
\end{enumerate}

\href{Casillas.html}{\emph{}}

anterior

Principio de las casillas

\href{Induccion.html}{}

siguiente

Principio de inducción

\emph{}

\emph{} Contenido

\begin{itemize}
\tightlist
\item
  \hyperref[introduccion]{Introducción}
\item
  \hyperref[principio-de-doble-conteo]{Principio de doble conteo}
\item
  \hyperref[coeficientes-binomiales-y-su-formula]{Coeficientes
  binomiales y su fórmula}
\item
  \hyperref[la-formula-de-pascal]{La fórmula de Pascal}
\item
  \hyperref[exploracion-computacional-del-triangulo-de-pascal]{Exploración
  computacional del triángulo de Pascal}
\item
  \hyperref[tarea-moral]{Tarea moral}
\end{itemize}

Por Leonardo Ignacio Martínez Sandoval

© Copyright 2022.\\

\end{document}
