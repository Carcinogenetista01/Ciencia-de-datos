\phantomsection\label{main-content}
{}

\emph{}

\begin{itemize}
\tightlist
\item
  \href{../_sources/P1/Induccion.ipynb}{{ \emph{} } {.ipynb}}
\item
  { \emph{} } {.pdf}
\end{itemize}

{ \emph{} }

{}

\phantomsection\label{jb-print-docs-body}
\section{Principio de inducción}\label{principio-de-inducciuxe3uxb3n}

\phantomsection\label{print-main-content}
\phantomsection\label{jb-print-toc}
\subsection{Contenido}\label{contenido}

\begin{itemize}
\tightlist
\item
  \hyperref[introduccion]{Introducción}
\item
  \hyperref[principio-de-induccion-basico]{Principio de inducción
  básico}
\item
  \hyperref[induccion-fuerte]{Inducción fuerte}
\item
  \hyperref[induccion-con-base-mas-grande]{Inducción con base más
  grande}
\item
  \hyperref[tarea-moral]{Tarea moral}
\end{itemize}

\phantomsection\label{searchbox}

\phantomsection\label{principio-de-induccion}
\section{\texorpdfstring{Principio de
inducción\hyperref[principio-de-induccion]{\#}}{Principio de inducción\#}}\label{principio-de-inducciuxe3uxb3n-1}

\phantomsection\label{introduccion}
\subsection{\texorpdfstring{Introducción\hyperref[introduccion]{\#}}{Introducción\#}}\label{introducciuxe3uxb3n}

En esta entrada supondremos que manejas el principio de inducción, por
lo menos a un nivel básico. En caso de no ser así, te recomendamos
revisar la siguiente entrada de blog:
\href{https://blog.nekomath.com/seminario-de-resolucion-de-problemas-principio-de-induccion/}{Introducción
a principio de inducción}.

Como quizás recuerdes, el principio de inducción permite hacer
demostraciones matemáticas para una cantidad infinita de números,
probando simplemente una cantidad finita de afirmaciones, usualmente
dos. En su versión más básica dice lo siguiente:

\textbf{Principio de inducción.} Para mostrar que una afirmación
{\textbackslash(P(n)\textbackslash)} es cierta para todo entero positivo
{\textbackslash(n\textbackslash)} basta con:

\begin{itemize}
\item
  Mostrar que {\textbackslash(P(1)\textbackslash)} es cierta.
\item
  Mostrar que para {\textbackslash(n\textbackslash geq
  1\textbackslash)}, se tiene que {\textbackslash(P(n)\textbackslash)}
  implica {\textbackslash(P(n+1)\textbackslash)}.
\end{itemize}

En esta entrada recordaremos cómo usar el principio de inducción en su
versión básica y luego discutiremos variantes del principio de
inducción que cumplen con el mismo objetivo, pero son más versátiles.

\phantomsection\label{principio-de-induccion-basico}
\subsection{\texorpdfstring{Principio de inducción
básico\hyperref[principio-de-induccion-basico]{\#}}{Principio de inducción básico\#}}\label{principio-de-inducciuxe3uxb3n-buxe3sico}

Para recordar cómo se usa el principio de inducción resolvamos el
siguiente problema.

\textbf{Problema.} Muestra que para todo entero
{\textbackslash(n\textbackslash)} se cumple que

\textbackslash{[}\textbackslash frac\{1\}\{1\textbackslash cdot 2\} +
\textbackslash frac\{1\}\{2\textbackslash cdot 3\} +
\textbackslash ldots + \textbackslash frac\{1\}\{n\textbackslash cdot
(n+1)\}=\textbackslash frac\{n\}\{n+1\}\textbackslash{]}

\emph{Solución.} Hagamos una solución que use principio de inducción.
Para ello, necesitamos comenzar verificando que la igualdad es cierta
cuando {\textbackslash(n=1\textbackslash)}. Luego, a partir de la
veracidad de la igualdad para cierta {\textbackslash(n\textbackslash)},
debemos demostrar la veracidad para {\textbackslash(n+1\textbackslash)}.

La veracidad para {\textbackslash(n=1\textbackslash)} es sencilla de
verificar, pues en lado izquierdo tendríamos únicamente al sumando
{\textbackslash(\textbackslash frac\{1\}\{1\textbackslash cdot 2\} =
\textbackslash frac\{1\}\{2\}\textbackslash)}, mientras que en el lado
derecho tendríamos la expresión
{\textbackslash(\textbackslash frac\{1\}\{2\}\textbackslash)}. Ambas
expresiones son iguales.

Como hipótesis inductiva, supongamos la veracidad de la igualdad para
cierto valor de {\textbackslash(n\textbackslash)}. Aquí
{\textbackslash(n\textbackslash)} es fijo, no estamos suponiendo la
veracidad para todo {\textbackslash(n\textbackslash)}, pues eso es lo
que queremos mostrar. De esta manera, tenemos que

\textbackslash{[}\textbackslash frac\{1\}\{1\textbackslash cdot 2\} +
\textbackslash frac\{1\}\{2\textbackslash cdot 3\} +
\textbackslash ldots + \textbackslash frac\{1\}\{n\textbackslash cdot
(n+1)\}=\textbackslash frac\{n\}\{n+1\}.\textbackslash{]}

Debemos mostrar la veracidad para {\textbackslash(n+2\textbackslash)},
es decir, debemos mostrar que

\textbackslash{[}\textbackslash frac\{1\}\{1\textbackslash cdot 2\} +
\textbackslash frac\{1\}\{2\textbackslash cdot 3\} +
\textbackslash ldots + \textbackslash frac\{1\}\{n\textbackslash cdot
(n+1)\} +
\textbackslash frac\{1\}\{(n+1)(n+2)\}=\textbackslash frac\{n+1\}\{n+2\}.\textbackslash{]}

Para ello, notemos que en el lado izquierdo aparece parte de la
expresión de la hipótesis inductiva. Tenemos entonces:

\textbackslash{[}\textbackslash begin\{align*\}
\textbackslash left(\textbackslash frac\{1\}\{1\textbackslash cdot 2\} +
\textbackslash frac\{1\}\{2\textbackslash cdot 3\} +
\textbackslash ldots + \textbackslash frac\{1\}\{n\textbackslash cdot
(n+1)\}\textbackslash right) + \textbackslash frac\{1\}\{(n+1)(n+2)\}
\&= \textbackslash frac\{n\}\{n+1\}+
\textbackslash frac\{1\}\{(n+1)(n+2)\}\textbackslash\textbackslash{}
\&=\textbackslash frac\{n(n+2)+(1)\}\{(n+1)(n+2)\}\textbackslash\textbackslash{}
\&=\textbackslash frac\{n\^{}2+2n+1\}\{(n+1)(n+2)\}\textbackslash\textbackslash{}
\&=\textbackslash frac\{(n+1)\^{}2\}\{(n+1)(n+2)\}\textbackslash\textbackslash{}
\&=\textbackslash frac\{n+1\}\{n+2\}.
\textbackslash end\{align*\}\textbackslash{]}

En la primera igualdad usamos la hipótesis inductiva. Luego,
simplemente hicimos las operaciones con fracciones. Notemos que llegamos
justo al lado derecho que queríamos llegar. Esto termina la prueba por
inducción. {{\textbackslash(\textbackslash square\textbackslash)}}

\phantomsection\label{induccion-fuerte}
\subsection{\texorpdfstring{Inducción
fuerte\hyperref[induccion-fuerte]{\#}}{Inducción fuerte\#}}\label{inducciuxe3uxb3n-fuerte}

Existen variantes del principio de inducción que son igual de válidas.
La siguiente se le conoce como el principio de inducción fuerte, pues
en la hipótesis inductiva suponemos no solamente un caso, sino también
todos los anteriores.

\textbf{Principio de inducción fuerte.} Para mostrar que una
afirmación {\textbackslash(P(n)\textbackslash)} es cierta para todo
entero positivo {\textbackslash(n\textbackslash)} basta con:

\begin{itemize}
\item
  Mostrar que {\textbackslash(P(1)\textbackslash)} es cierta.
\item
  Mostrar que para {\textbackslash(n\textbackslash geq
  2\textbackslash)}, \textbf{todas} las afirmaciones
  {\textbackslash(P(k)\textbackslash)} para
  {\textbackslash(k\textless n\textbackslash)} implican en conjunto
  {\textbackslash(P(n)\textbackslash)}.
\end{itemize}

Veamos un problema interesante que hace uso de esta versión del
principio de inducción.

\textbf{Problema.} Muestra que cualquier número entero positivo se
puede expresar como suma de números de Fibonacci distintos.

\emph{Solución.} Procedemos por inducción fuerte. En el caso base
tenemos a {\textbackslash(1\textbackslash)}, que ya es por sí mismo un
número de Fibonacci, así que tiene una expresión como la que buscamos.

Tomemos ahora un entero {\textbackslash(n\textbackslash geq
2\textbackslash)} y supongamos que todos los números desde
{\textbackslash(1\textbackslash)} hasta
{\textbackslash(n-1\textbackslash)} se pueden escribir como suma de
Fibonacci distintos. Tomemos el entero {\textbackslash(k\textbackslash)}
más grande posible tal que
{\textbackslash(F\_k\textless n\textbackslash)}. Como
{\textbackslash(k\textbackslash)} es el entero más grande para el cual
esto pasa, tenemos que {\textbackslash(n\textbackslash leq
F\_\{k+1\}=F\_\{k\}+F\_\{k-1\}\textbackslash)}, de donde

\textbackslash{[}n-F\_k\textbackslash leq F\_\{k-1\}.\textbackslash{]}

Como {\textbackslash(n\textbackslash)} es mayor que
{\textbackslash(F\_k\textbackslash)}, se tiene que
{\textbackslash(n-F\_k\textbackslash geq 1\textbackslash)}. Además,
{\textbackslash(n-F\_k\textbackslash lt n\textbackslash)}. Por
hipótesis inductiva, el número {\textbackslash(n-F\_k\textbackslash)}
se debe poder expresar como suma de números de Fibonacci distintos,
digamos

\textbackslash{[}n-F\_k=F\_\{i\_1\}+F\_\{i\_2\}+\textbackslash ldots+F\_\{i\_r\}.\textbackslash{]}

Como

\textbackslash{[}n-F\_k\textbackslash leq F\_\{k-1\},\textbackslash{]}

ninguno de los sumandos puede exceder a
{\textbackslash(F\_\{k-1\}\textbackslash)}. Así, todos ellos son menores
a {\textbackslash(F\_k\textbackslash)}. Por lo tanto, tenemos que

\textbackslash{[}n=F\_\{i\_1\}+F\_\{i\_2\}+\textbackslash ldots+F\_\{i\_r\}+F\_k,\textbackslash{]}

en donde en el lado derecho tenemos números de Fibonacci distintos,
como queríamos. {{\textbackslash(\textbackslash square\textbackslash)}}

La idea de usar inducción fuerte es muy buena: podemos demostrar cosas
utilizando casos que ya hemos demostrado. Esta misma intuición la
utilizaremos cuando diseñemos algoritmos recursivos. Mientras tanto,
estudia el siguiente código para ver cómo llevamos la idea de la
demostración anterior a un algoritmo que escribe a cualquier entero
positivo como suma de Fibonaccis distintos.

\begin{verbatim}
def sum_fibo(n):
    fibo=[0,1]
    while fibo[-1]<n:
        fibo.append(fibo[-1]+fibo[-2])
    sumandos=[]
    suma=0
    for j in fibo[::-1]:
        if suma+j<=n:
            sumandos.append(j)
            suma+=j
            if suma==n:
                break

    print("El número {} es la suma de los Fibonaccis en la lista {}".format(n,sumandos))

sum_fibo(553)
sum_fibo(53)
sum_fibo(111)
\end{verbatim}

\begin{verbatim}
El número 553 es la suma de los Fibonaccis en la lista [377, 144, 21, 8, 3]
El número 53 es la suma de los Fibonaccis en la lista [34, 13, 5, 1]
El número 111 es la suma de los Fibonaccis en la lista [89, 21, 1]
\end{verbatim}

\phantomsection\label{induccion-con-base-mas-grande}
\subsection{\texorpdfstring{Inducción con base más
grande\hyperref[induccion-con-base-mas-grande]{\#}}{Inducción con base más grande\#}}\label{inducciuxe3uxb3n-con-base-muxe3s-grande}

Cuando en la demostración del paso inductivo usamos más de un elemento
anterior, es importante que en los casos base revisemos tantos casos
como la demostración de la hipótesis inductiva requiera. Para ver un
ejemplo de esto, retomemos un ejemplo que dejamos pendiente acerca de
cómo son los residuos al dividir entre
{\textbackslash(3\textbackslash)} de los números de Fibonacci.

\textbf{Problema.} Los residuos de la sucesión de Fibonacci al dividir
entre {\textbackslash(3\textbackslash)} se ciclan, con un ciclo de
periodo {\textbackslash(8\textbackslash)} que es
{\textbackslash(0,1,1,2,0,2,2,1\textbackslash)}.

\emph{Demostración.} Hacemos el caso base de manera computacional,
calculando los primeros {\textbackslash(8\textbackslash)} números de la
sucesión de Fibonacci y verificando que en efecto se cumple lo que
decimos:

\begin{verbatim}
a,b=0,1
for j in range(8):
    print("Para {} el Fibonacci es {} y deja residuo {} al dividirse entre 3".format(j,a,a%3))
    a,b=b,a+b
\end{verbatim}

\begin{verbatim}
Para 0 el Fibonacci es 0 y deja residuo 0 al dividirse entre 3
Para 1 el Fibonacci es 1 y deja residuo 1 al dividirse entre 3
Para 2 el Fibonacci es 1 y deja residuo 1 al dividirse entre 3
Para 3 el Fibonacci es 2 y deja residuo 2 al dividirse entre 3
Para 4 el Fibonacci es 3 y deja residuo 0 al dividirse entre 3
Para 5 el Fibonacci es 5 y deja residuo 2 al dividirse entre 3
Para 6 el Fibonacci es 8 y deja residuo 2 al dividirse entre 3
Para 7 el Fibonacci es 13 y deja residuo 1 al dividirse entre 3
\end{verbatim}

Ahora demostraremos el resultado «en bloques de
{\textbackslash(8\textbackslash)}». Así, para cierta
{\textbackslash(n\textbackslash geq 1\textbackslash)}, supongamos que el
resultado es cierto para los primeros {\textbackslash(8n\textbackslash)}
números de Fibonacci (hasta
{\textbackslash(F\_\{8n-1\}\textbackslash)}). Lo que haremos en el paso
inductivo es ver que para los siguientes
{\textbackslash(8\textbackslash)} números de Fibonacci también es
cierto.

Comencemos viendo que esto sucede para
{\textbackslash(F\_\{8n\}\textbackslash)}, donde tenemos que ver que el
residuo es {\textbackslash(0\textbackslash)}.Usamos que que:

\textbackslash{[}F\_\{8n\}=F\_\{8n-1\}+F\_\{8n-2\}\textbackslash{]}

Como por hipótesis inductiva tenemos que
{\textbackslash(F\_\{8n-1\}\textbackslash)} deja residuo
{\textbackslash(1\textbackslash)} al dividirse entre
{\textbackslash(3\textbackslash)} y
{\textbackslash(F\_\{8n-2\}\textbackslash)} deja residuo
{\textbackslash(2\textbackslash)} al dividirse entre
{\textbackslash(3\textbackslash)}, tenemos que
{\textbackslash(F\_\{8n\}\textbackslash)} es múltiplo de
{\textbackslash(3\textbackslash)}, como queríamos. En notación de
congruencias, podemos escribir esto así:

\textbackslash{[}F\_\{8n\}=F\_\{8n-1\}+F\_\{8n-2\}\textbackslash equiv 1
+ 2 \textbackslash equiv 0 \textbackslash pmod\{3\}.\textbackslash{]}

De manera similar,

\textbackslash{[}\textbackslash begin\{align*\}
F\_\{8n+1\}\&=F\_\{8n\}+F\_\{8n-1\}\textbackslash equiv 0 + 1
\textbackslash equiv 1
\textbackslash pmod\{3\}\textbackslash\textbackslash{}
F\_\{8n+2\}\&=F\_\{8n+1\}+F\_\{8n\}\textbackslash equiv 1 + 0
\textbackslash equiv 1
\textbackslash pmod\{3\}\textbackslash\textbackslash{}
F\_\{8n+3\}\&=F\_\{8n+2\}+F\_\{8n+1\}\textbackslash equiv 1 + 1
\textbackslash equiv 2
\textbackslash pmod\{3\}\textbackslash\textbackslash{}
F\_\{8n+4\}\&=F\_\{8n+3\}+F\_\{8n+2\}\textbackslash equiv 2 + 1
\textbackslash equiv 0
\textbackslash pmod\{3\}\textbackslash\textbackslash{}
F\_\{8n+5\}\&=F\_\{8n+4\}+F\_\{8n+3\}\textbackslash equiv 0 + 2
\textbackslash equiv 2
\textbackslash pmod\{3\}\textbackslash\textbackslash{}
F\_\{8n+6\}\&=F\_\{8n+5\}+F\_\{8n+4\}\textbackslash equiv 2 + 0
\textbackslash equiv 2
\textbackslash pmod\{3\}\textbackslash\textbackslash{}
F\_\{8n+7\}\&=F\_\{8n+6\}+F\_\{8n+5\}\textbackslash equiv 2 + 2
\textbackslash equiv 1
\textbackslash pmod\{3\}\textbackslash\textbackslash{}
\textbackslash end\{align*\}\textbackslash{]}

Esto termina la prueba del paso inductivo y por lo tanto la
demostración. {{\textbackslash(\textbackslash square\textbackslash)}}

\subsection{\texorpdfstring{Tarea
moral\hyperref[tarea-moral]{\#}}{Tarea moral\#}}\label{tarea-moral}

Los siguientes problemas te ayudarán a practicar lo visto en esta
entrada. Para resolverlos, necesitarás usar herramientas matemáticas,
computacionales o ambas.

\begin{enumerate}
\item
  Si no somos cuidadosos con los argumentos inductivos, podemos tener
  problemas y demostrar cosas falsas. Considera el siguiente
  «argumento» inductivo que muestra que todos los gatos son del mismo
  color. Procedemos por inducción. Si tenemos un gato, sólo hay un
  color que verificar. Supongamos que ya verificamos que siempre que
  tenemos {\textbackslash(n\textbackslash)} gatos, entonces todos son
  del mismo color. Toma un conjunto {\textbackslash(X\textbackslash)} de
  {\textbackslash(n+1\textbackslash)} gatos y dos gatos
  {\textbackslash(G\textbackslash)} y {\textbackslash(H\textbackslash)}
  de {\textbackslash(X\textbackslash)}. Si a
  {\textbackslash(X\textbackslash)} le quitamos el gato
  {\textbackslash(G\textbackslash)}, por hipótesis inductiva todos son
  del mismo color. Si a {\textbackslash(X\textbackslash)} le quitamos el
  gato {\textbackslash(H\textbackslash)}, por hipótesis todos son del
  mismo color. Así, {\textbackslash(G\textbackslash)} es del mismo color
  que los gatos de {\textbackslash(X\textbackslash)}, que son del mismo
  color que {\textbackslash(H\textbackslash)}, así que ya todos los
  {\textbackslash(n+1\textbackslash)} gatos fueron del mismo color. Esto
  termina la «demostración». ¿Cuál es el problema?
\item
  Considera la matriz {\textbackslash(A=\textbackslash begin\{pmatrix\}
  1 \& 2 \textbackslash\textbackslash{} 0 \&
  1\textbackslash end\{pmatrix\}\textbackslash)}. Encuentra a mano las
  matrices {\textbackslash(A\^{}2\textbackslash)},
  {\textbackslash(A\^{}3\textbackslash)},
  {\textbackslash(A\^{}4\textbackslash)},
  {\textbackslash(A\^{}5\textbackslash)}. Obsérvalas con detenimiento.
  Conjetura una expresión cerrada para la matriz
  {\textbackslash(A\^{}n\textbackslash)} y demuéstrala por inducción.
\item
  Regresa al capítulo de «Buscar un patrón», a la sección de
  «Triángulo de Pascal con saltos de tres en tres». Demuestra por
  inducción todas las observaciones que se hicieron para obtener la
  solución.
\item
  Muestra que cualquier número entero positivo se puede expresar de
  manera única como suma de números de Fibonacci distintos, en donde
  ademas no usamos dos números de Fibonacci
  {\textbackslash(F\_n\textbackslash)} y
  {\textbackslash(F\_m\textbackslash)} con
  {\textbackslash(n\textbackslash)} y {\textbackslash(m\textbackslash)}
  consecutivos.
\item
  Hay otra variante del principio de inducción que se llama
  «inducción de Cauchy». Revisa la siguiente entrada de blog para
  averiguar en qué consiste:
  \href{https://blog.nekomath.com/seminario-de-resolucion-de-problemas-principio-de-induccion-parte-3/}{Variantes
  del principio de inducción}.
\end{enumerate}

\href{DobleConteo.html}{\emph{}}

anterior

Principio de doble conteo y coeficientes binomiales

\href{Recursion.html}{}

siguiente

Principio de recursión

\emph{}

\emph{} Contenido

\begin{itemize}
\tightlist
\item
  \hyperref[introduccion]{Introducción}
\item
  \hyperref[principio-de-induccion-basico]{Principio de inducción
  básico}
\item
  \hyperref[induccion-fuerte]{Inducción fuerte}
\item
  \hyperref[induccion-con-base-mas-grande]{Inducción con base más
  grande}
\item
  \hyperref[tarea-moral]{Tarea moral}
\end{itemize}

Por Leonardo Ignacio Martínez Sandoval

© Copyright 2022.\\
