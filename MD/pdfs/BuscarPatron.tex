% Options for packages loaded elsewhere
\PassOptionsToPackage{unicode}{hyperref}
\PassOptionsToPackage{hyphens}{url}
\documentclass[
]{article}
\usepackage{xcolor}
\usepackage[margin=1in]{geometry}
\usepackage{amsmath,amssymb}
\setcounter{secnumdepth}{-\maxdimen} % remove section numbering
\usepackage{iftex}
\ifPDFTeX
  \usepackage[T1]{fontenc}
  \usepackage[utf8]{inputenc}
  \usepackage{textcomp} % provide euro and other symbols
\else % if luatex or xetex
  \usepackage{unicode-math} % this also loads fontspec
  \defaultfontfeatures{Scale=MatchLowercase}
  \defaultfontfeatures[\rmfamily]{Ligatures=TeX,Scale=1}
\fi
\usepackage{lmodern}
\ifPDFTeX\else
  % xetex/luatex font selection
    \setmainfont[]{DejaVu Serif}
    \setmonofont[]{DejaVu Sans Mono}
\fi
% Use upquote if available, for straight quotes in verbatim environments
\IfFileExists{upquote.sty}{\usepackage{upquote}}{}
\IfFileExists{microtype.sty}{% use microtype if available
  \usepackage[]{microtype}
  \UseMicrotypeSet[protrusion]{basicmath} % disable protrusion for tt fonts
}{}
\makeatletter
\@ifundefined{KOMAClassName}{% if non-KOMA class
  \IfFileExists{parskip.sty}{%
    \usepackage{parskip}
  }{% else
    \setlength{\parindent}{0pt}
    \setlength{\parskip}{6pt plus 2pt minus 1pt}}
}{% if KOMA class
  \KOMAoptions{parskip=half}}
\makeatother
\usepackage{longtable,booktabs,array}
\usepackage{calc} % for calculating minipage widths
% Correct order of tables after \paragraph or \subparagraph
\usepackage{etoolbox}
\makeatletter
\patchcmd\longtable{\par}{\if@noskipsec\mbox{}\fi\par}{}{}
\makeatother
% Allow footnotes in longtable head/foot
\IfFileExists{footnotehyper.sty}{\usepackage{footnotehyper}}{\usepackage{footnote}}
\makesavenoteenv{longtable}
\ifLuaTeX
\usepackage[bidi=basic]{babel}
\else
\usepackage[bidi=default]{babel}
\fi
\babelprovide[main,import]{spanish}
\ifPDFTeX
\else
\babelfont{rm}[]{DejaVu Serif}
\fi
% get rid of language-specific shorthands (see #6817):
\let\LanguageShortHands\languageshorthands
\def\languageshorthands#1{}
\setlength{\emergencystretch}{3em} % prevent overfull lines
\providecommand{\tightlist}{%
  \setlength{\itemsep}{0pt}\setlength{\parskip}{0pt}}
\usepackage{bookmark}
\IfFileExists{xurl.sty}{\usepackage{xurl}}{} % add URL line breaks if available
\urlstyle{same}
\hypersetup{
  pdftitle={Buscar un patrón --- Matemáticas Discretas para Ciencia de Datos},
  pdflang={es},
  hidelinks,
  pdfcreator={LaTeX via pandoc}}

\title{Buscar un patrón --- Matemáticas Discretas para Ciencia de
Datos}
\author{}
\date{}


        \usepackage{fontspec}
        \usepackage{unicode-math}
        \setmainfont{DejaVu Serif}
        \setmonofont{DejaVu Sans Mono}
        \usepackage{amsmath}
        \usepackage{amssymb}
        \usepackage{graphicx}
        \usepackage{hyperref}
        \usepackage[spanish]{babel}
        
\begin{document}
\maketitle

\phantomsection\label{main-content}
{}

\emph{}

\begin{itemize}
\tightlist
\item
  \href{../_sources/P1/BuscarPatron.ipynb}{{ \emph{} } {.ipynb}}
\item
  { \emph{} } {.pdf}
\end{itemize}

{ \emph{} }

{}

\phantomsection\label{jb-print-docs-body}
\section{Buscar un patrón}\label{buscar-un-patruxe3uxb3n}

\phantomsection\label{print-main-content}
\phantomsection\label{jb-print-toc}
\subsection{Contenido}\label{contenido}

\begin{itemize}
\tightlist
\item
  \hyperref[introduccion]{Introducción}
\item
  \hyperref[fibonaccis-y-residuos-que-dejan]{Fibonaccis y residuos que
  dejan}
\item
  \hyperref[suma-por-renglones-en-el-triangulo-de-pascal]{Suma por
  renglones en el triángulo de Pascal}
\item
  \hyperref[triangulo-de-pascal-con-saltos-de-tres-en-tres]{Triángulo
  de Pascal con saltos de tres en tres}
\item
  \hyperref[tarea-moral]{Tarea moral}
\end{itemize}

\phantomsection\label{searchbox}

\phantomsection\label{buscar-un-patron}
\section{\texorpdfstring{Buscar un
patrón\hyperref[buscar-un-patron]{\#}}{Buscar un patrón\#}}\label{buscar-un-patruxe3uxb3n-1}

\phantomsection\label{introduccion}
\subsection{\texorpdfstring{Introducción\hyperref[introduccion]{\#}}{Introducción\#}}\label{introducciuxe3uxb3n}

La heurística de buscar un patrón consiste en tomar un problema,
resolver casos pequeños o iniciales, y descubrir algo que esté
pasando. Esta es una idea muy general, pero puede ayudar a descubrir
varias cosas. Por ejemplo, puede que:

\begin{itemize}
\item
  Tras hacer algunos casos veamos una fórmula.
\item
  Después de jugar un poco con el problema, veamos que hay un ciclo que
  aparece.
\item
  Al hacer algunas iteraciones de un problema, veamos que ciertos
  números siempre son crecientes.
\item
  Etcétera.
\end{itemize}

El encontrar un patrón puede a veces dar de manera inmediata la
solución que estamos buscando. Sin embargo, en la mayoría de los casos
es simplemente un paso intermedio dentro de una solución más compleja.

La exploración del problema puede ser hecha a mano o de manera
computacional. En las siguientes secciones veremos algunos ejemplos de
esto en acción.

\subsection{\texorpdfstring{Fibonaccis y residuos que
dejan\hyperref[fibonaccis-y-residuos-que-dejan]{\#}}{Fibonaccis y residuos que dejan\#}}\label{fibonaccis-y-residuos-que-dejan}

La sucesión de Fibonacci es una de las más sencillas de definir.
Además, rapidamente encontramos patrones interesantes en ella. Es la
sucesión que comienza con {\textbackslash(0\textbackslash)}, luego
{\textbackslash(1\textbackslash)} y a partir de ahí cada número es la
suma de los dos anteriores. Así, los primeros números de la sucesión
de Fibonacci son:

\textbackslash{[}0,1,1,2,3,5,8,13,21,\textbackslash ldots\textbackslash{]}

En símbolos, es la sucesión {\textbackslash(F\_n\textbackslash)} tal
que {\textbackslash(F\_0=0\textbackslash)},
{\textbackslash(F\_1=1\textbackslash)} y para
{\textbackslash(n\textbackslash geq 0\textbackslash)} cumple que
{\textbackslash(F\_\{n+2\}=F\_n+F\_\{n+1\}\textbackslash)}. Consideremos
el siguiente problema.

\textbf{Problema.} Sea
{\textbackslash(\textbackslash\{F\_n\textbackslash\}\textbackslash)} la
sucesión de Fibonacci.

\begin{itemize}
\item
  Encuentra para cuántos enteros {\textbackslash(n\textbackslash)} en
  {\textbackslash(\textbackslash\{0,1,2,\textbackslash ldots,100\textbackslash\}\textbackslash)}
  se cumple que {\textbackslash(F\_n\textbackslash)} es un múltiplo de
  {\textbackslash(3\textbackslash)}.
\item
  Encuentra para cuántos enteros {\textbackslash(n\textbackslash)} en
  {\textbackslash(\textbackslash\{0,1,2,\textbackslash ldots,1000\textbackslash\}\textbackslash)}
  se cumple que {\textbackslash(F\_n\textbackslash)} es un múltiplo de
  {\textbackslash(5\textbackslash)}.
\end{itemize}

\emph{Solución.} Comencemos con el primer inciso. Lo que podríamos
hacer es comenzar a escribir a mano algunos números de Fibonacci y ver
si encontramos algún patrón evidente:
{\textbackslash(0,1,1,2,3,5,8,13,21,\textbackslash ldots\textbackslash)}.

Hasta aquí, parece ser que los números de Fibonacci que son múltiplos
de {\textbackslash(3\textbackslash)} son
{\textbackslash(F\_0\textbackslash)} y
{\textbackslash(F\_4\textbackslash)}. Esto es muy poca evidencia para
hacer una conjetura decente. Hagamos más casos.

En vez de realizar esta labor a mano, podemos pedirle a la computadora
que haga muchos casos más. El siguiente es un bloque de código en
Python. Tras ejecutarlo, muestra los primeros 15 números de Fibonacci y
su residuo al dividirse entre {\textbackslash(3\textbackslash)}. Por el
momento no es tan importante por qué el código funciona. Más adelante
platicaremos de cómo se nos puede ocurrir esto.

\begin{verbatim}
a,b=0,1

for j in range(15):
    print("Para n={}, Fn es {}. Su residuo al dividir entre 3 es {},".format(j,a,a%3))
    a,b=b,a+b
\end{verbatim}

\begin{verbatim}
Para n=0, Fn es 0. Su residuo al dividir entre 3 es 0,
Para n=1, Fn es 1. Su residuo al dividir entre 3 es 1,
Para n=2, Fn es 1. Su residuo al dividir entre 3 es 1,
Para n=3, Fn es 2. Su residuo al dividir entre 3 es 2,
Para n=4, Fn es 3. Su residuo al dividir entre 3 es 0,
Para n=5, Fn es 5. Su residuo al dividir entre 3 es 2,
Para n=6, Fn es 8. Su residuo al dividir entre 3 es 2,
Para n=7, Fn es 13. Su residuo al dividir entre 3 es 1,
Para n=8, Fn es 21. Su residuo al dividir entre 3 es 0,
Para n=9, Fn es 34. Su residuo al dividir entre 3 es 1,
Para n=10, Fn es 55. Su residuo al dividir entre 3 es 1,
Para n=11, Fn es 89. Su residuo al dividir entre 3 es 2,
Para n=12, Fn es 144. Su residuo al dividir entre 3 es 0,
Para n=13, Fn es 233. Su residuo al dividir entre 3 es 2,
Para n=14, Fn es 377. Su residuo al dividir entre 3 es 2,
\end{verbatim}

Aquí parece ser mucho más claro cuándo un número de Fibonacci es
múltiplo de {\textbackslash(3\textbackslash)}. De acuerdo a lo
anterior, los números de Fibonacci que son múltiplos de
{\textbackslash(3\textbackslash)} son
{\textbackslash(F\_0\textbackslash)},
{\textbackslash(F\_4\textbackslash)},
{\textbackslash(F\_8\textbackslash)} y
{\textbackslash(F\_\{12\}\textbackslash)}, de donde ahora tenemos más
evidencia para conjeturar lo siguiente.

\textbf{Conjetura.} El Fibonacci {\textbackslash(F\_n\textbackslash)} es
múltiplo de {\textbackslash(3\textbackslash)} si y sólo si
{\textbackslash(n\textbackslash)} es múltiplo de
{\textbackslash(4\textbackslash)}.

De hecho, la exploración nos permite hacer una conjetura mucho más
fuerte que nos dice qué residuos van dejando los números de Fibonacci
al dividirse entre {\textbackslash(3\textbackslash)}.

\textbf{Conjetura.} Los residuos de la sucesión de Fibonacci al dividir
entre {\textbackslash(3\textbackslash)} se ciclan, con un ciclo de
periodo {\textbackslash(8\textbackslash)} que es
{\textbackslash(0,1,1,2,0,2,2,1\textbackslash)}.

Podemos usar esta conjetura para sospechar cuál es la respuesta al
problema: hay {\textbackslash(26\textbackslash)} enteros
{\textbackslash(n\textbackslash)} en el conjunto
{\textbackslash(\textbackslash\{0,1,2,\textbackslash ldots,100\textbackslash\}\textbackslash)}
tales que {\textbackslash(F\_n\textbackslash)} es múltiplo de
{\textbackslash(3\textbackslash)}, a saber cada uno de los múltiplos de
{\textbackslash(4\textbackslash)} en ese conjunto.

Por supuesto, para estar totalmente seguros de esta respuesta tendríamos
que verificar que en efecto el patrón se cumple. Más adelante, cuando
repasemos el principio de inducción, retomaremos este problema y
daremos una demostración formal de que los residuos se ciclan así.

La segunda parte del problema queda para que tú realices una conjetura.
{{\textbackslash(\textbackslash square\textbackslash)}}

\phantomsection\label{suma-por-renglones-en-el-triangulo-de-pascal}
\subsection{\texorpdfstring{Suma por renglones en el triángulo de
Pascal\hyperref[suma-por-renglones-en-el-triangulo-de-pascal]{\#}}{Suma por renglones en el triángulo de Pascal\#}}\label{suma-por-renglones-en-el-triuxe3ngulo-de-pascal}

Otro objeto matemático en el cual aparecen varios patrones interesantes
es el triángulo de Pascal. A continuación se muestran sus primeros
renglones.

\begin{longtable}[]{@{}llllllllllll@{}}
\toprule\noalign{}
& & & & & & & & & & & \\
\midrule\noalign{}
\endhead
\bottomrule\noalign{}
\endlastfoot
& & & & & 1 & & & & & & \\
& & & & 1 & & 1 & & & & & \\
& & & 1 & & 2 & & 1 & & & & \\
& & 1 & & 3 & & 3 & & 1 & & & \\
& 1 & & 4 & & 6 & & 4 & & 1 & & \\
1 & & 5 & & 10 & & 10 & & 5 & & 1 & \\
\end{longtable}

El triángulo de Pascal está construido como sigue. Al inicio, se
coloca un uno hasta arriba. Luego, se van agregando renglones, cada vez
con un número más que el renglón anterior, con las siguientes reglas:

\begin{itemize}
\item
  Si la posición está en los extremos, el número que se agrega es
  {\textbackslash(1\textbackslash)}.
\item
  Si la posición es intermedia, entonces es la suma de los dos números
  que tenga arriba.
\end{itemize}

Así, por ejemplo, nota que el {\textbackslash(10\textbackslash)} en el
sexto renglón del triángulo es la suma de los dos números que tiene
arriba, es decir {\textbackslash(10=4+6\textbackslash)}. A partir de
estas nociones básicas, podemos hacernos una pregunta con respecto a la
suma de los elementos de cada renglón.

\textbf{Pregunta.} ¿Cuál es el primer renglón del triángulo de
Pascal en el cual la suma de todos sus elementos es mayor o igual a un
millón?

\emph{Solución.} El problema nos invita a explorar cómo es la suma de
los elementos en cada uno de los renglones del triángulo de Pascal.
Coloquemos dicha suma hasta la derecha de cada renglón. Por ejemplo,
como en el cuarto renglón la suma de los elementos es
{\textbackslash(1+3+3+1=8\textbackslash)}, en la siguiente figura hemos
puesto a {\textbackslash(8\textbackslash)} hasta la derecha, en
negritas.

\begin{longtable}[]{@{}llllllllllll@{}}
\toprule\noalign{}
& & & & & & & & & & & \\
\midrule\noalign{}
\endhead
\bottomrule\noalign{}
\endlastfoot
& & & & & 1 & & & & & & \textbf{1} \\
& & & & 1 & & 1 & & & & & \textbf{2} \\
& & & 1 & & 2 & & 1 & & & & \textbf{4} \\
& & 1 & & 3 & & 3 & & 1 & & & \textbf{8} \\
& 1 & & 4 & & 6 & & 4 & & 1 & & \textbf{16} \\
1 & & 5 & & 10 & & 10 & & 5 & & 1 & \textbf{32} \\
\end{longtable}

Aquí hay un patrón evidente (¿cuál?). Si quisiéremos obtener más
evidencia de que este patrón sigue, podríamos pedirle a Python que
hiciera más renglones del triángulo de Pascal, por ejemplo, que haga
los primeros {\textbackslash(10\textbackslash)} renglones. Esto se logra
mediante el siguiente código. Una vez más, puedes estudiar el código
pero a estas alturas no es totalmente necesario que entiendas a
profundidad qué está haciendo.

\begin{verbatim}
renglon=[1]
print("El renglón {} tiene suma {}".format(renglon,sum(renglon)))

for j in range(1,10):
    new_renglon=[]
    for k in range(j+1):
        if k==0 or k==j:
            new_renglon.append(1)
        else:
            new_renglon.append(renglon[k-1]+renglon[k])
    renglon=new_renglon
    print("El renglón {} tiene suma {}".format(renglon,sum(renglon)))
\end{verbatim}

\begin{verbatim}
El renglón [1] tiene suma 1
El renglón [1, 1] tiene suma 2
El renglón [1, 2, 1] tiene suma 4
El renglón [1, 3, 3, 1] tiene suma 8
El renglón [1, 4, 6, 4, 1] tiene suma 16
El renglón [1, 5, 10, 10, 5, 1] tiene suma 32
El renglón [1, 6, 15, 20, 15, 6, 1] tiene suma 64
El renglón [1, 7, 21, 35, 35, 21, 7, 1] tiene suma 128
El renglón [1, 8, 28, 56, 70, 56, 28, 8, 1] tiene suma 256
El renglón [1, 9, 36, 84, 126, 126, 84, 36, 9, 1] tiene suma 512
\end{verbatim}

Con esto podemos realizar una conjetura. Para que nuestra conjetura
quede un poco más simple, pensemos en que el renglón de hasta arriba
es el renglón {\textbackslash(0\textbackslash)}, el siguiente es el
renglón {\textbackslash(1\textbackslash)} y así sucesivamente.

\textbf{Conjetura.} La suma de los números en el renglón
{\textbackslash(n\textbackslash)} del triángulo de Pascal es
{\textbackslash(2\^{}n\textbackslash)}.

Más adelante, cuando hablemos de doble conteo, tendremos más
herramientas para demostrar esto. Sin embargo, de momento tomémoslo
como un hecho y usémoslo para responder el problema. Si queremos que
los números de un renglón excedan un millón, entonces necesitamos que
{\textbackslash(2\^{}n\textbackslash geq 1000000\textbackslash)}.
Aprovechemos que podemos usar Python aquí para saber cuándo sucede
esto.

\begin{verbatim}
n=0
while 2**n<1000000:
    n+=1
print(n)
\end{verbatim}

\begin{verbatim}
20
\end{verbatim}

Así, la primera vez que {\textbackslash(2\^{}n\textbackslash)} excede un
millón (y por lo tanto la respuesta a nuestro problema) es con
{\textbackslash(n=20\textbackslash)}.
{{\textbackslash(\textbackslash square\textbackslash)}}

Hasta el final de la solución pudimos haber simplemente copiado el
código anterior para que Python hiciera la suma de nuevo. Esto no sería
ningún problema, y lo haría muy rápido. Sin embargo, de manera
intuitiva es fácil convencerse de que gracias a la conjetura se puede
plantear el nuevo código y con él se hacen «muchas menos
operaciones». Más adelante, cuando hablemos de complejidad
computacional, formalizaremos esto.

\phantomsection\label{triangulo-de-pascal-con-saltos-de-tres-en-tres}
\subsection{\texorpdfstring{Triángulo de Pascal con saltos de tres en
tres\hyperref[triangulo-de-pascal-con-saltos-de-tres-en-tres]{\#}}{Triángulo de Pascal con saltos de tres en tres\#}}\label{triuxe3ngulo-de-pascal-con-saltos-de-tres-en-tres}

En los problemas anteriores el patrón que debemos encontrar sale muy
rápido. Esto no necesariamente será el caso cuando tengamos problemas
más complicados. Veamos un problema en el que la exploración y los
patrones que debemos encontrar son mucho más elaborados.

\textbf{Problema.} ¿Cuánto suman los elementos del renglón
{\textbackslash(100\textbackslash)} del triángulo de Pascal si los
sumamos comenzando con el primero y saltando de tres en tres?

El problema sugiere ver en cada renglón cuánto suman los elementos si
saltamos de tres en tres. Es decir, para los primeros renglones del
triángulo de Pascal la siguiente figura muestra en negritas los
números que tendríamos que sumar en cada renglón.

\begin{longtable}[]{@{}llllllllllll@{}}
\toprule\noalign{}
& & & & & & & & & & & \\
\midrule\noalign{}
\endhead
\bottomrule\noalign{}
\endlastfoot
& & & & & \textbf{1} & & & & & & \\
& & & & \textbf{1} & & 1 & & & & & \\
& & & \textbf{1} & & 2 & & 1 & & & & \\
& & \textbf{1} & & 3 & & 3 & & \textbf{1} & & & \\
& \textbf{1} & & 4 & & 6 & & \textbf{4} & & 1 & & \\
\textbf{1} & & 5 & & 10 & & \textbf{10} & & 5 & & 1 & \\
\end{longtable}

Específicamente, queremos ver qué sucede en el renglón
{\textbackslash(100\textbackslash)}.

\emph{Solución.} Si comenzamos a explorar el problema de manera
directa, es muy probable que no encontremos un patrón de manera
inmediata para la suma de los elementos que nos interesan. Llamando
{\textbackslash(A\_n\textbackslash)} a la suma de los elementos del
renglón {\textbackslash(n\textbackslash)} saltando de
{\textbackslash(3\textbackslash)} en {\textbackslash(3\textbackslash)}
número que nos interesa, hasta el momento tenemos la siguiente tabla:

\begin{longtable}[]{@{}ll@{}}
\toprule\noalign{}
Valor de {\textbackslash(n\textbackslash)} & Valor de
{\textbackslash(A\_n\textbackslash)} \\
\midrule\noalign{}
\endhead
\bottomrule\noalign{}
\endlastfoot
0 & 1 \\
1 & 1 \\
2 & 1 \\
3 & 2 \\
4 & 5 \\
5 & 11 \\
\end{longtable}

Aquí no hay nada obvio sucediendo. ¿Qué sucede si le pedimos a Python
que obtenga más valores?

\begin{verbatim}
renglon=[1]
print("El renglón {} tiene suma {}".format(renglon,sum(renglon)))

for j in range(1,15):
    new_renglon=[]
    for k in range(j+1):
        if k==0 or k==j:
            new_renglon.append(1)
        else:
            new_renglon.append(renglon[k-1]+renglon[k])
    renglon=new_renglon
    mult_3=renglon[0::3]
    print("El renglón {} saltando de 3 en 3 tiene suma {}".format(renglon,sum(mult_3)))
\end{verbatim}

\begin{verbatim}
El renglón [1] tiene suma 1
El renglón [1, 1] saltando de 3 en 3 tiene suma 1
El renglón [1, 2, 1] saltando de 3 en 3 tiene suma 1
El renglón [1, 3, 3, 1] saltando de 3 en 3 tiene suma 2
El renglón [1, 4, 6, 4, 1] saltando de 3 en 3 tiene suma 5
El renglón [1, 5, 10, 10, 5, 1] saltando de 3 en 3 tiene suma 11
El renglón [1, 6, 15, 20, 15, 6, 1] saltando de 3 en 3 tiene suma 22
El renglón [1, 7, 21, 35, 35, 21, 7, 1] saltando de 3 en 3 tiene suma 43
El renglón [1, 8, 28, 56, 70, 56, 28, 8, 1] saltando de 3 en 3 tiene suma 85
El renglón [1, 9, 36, 84, 126, 126, 84, 36, 9, 1] saltando de 3 en 3 tiene suma 170
El renglón [1, 10, 45, 120, 210, 252, 210, 120, 45, 10, 1] saltando de 3 en 3 tiene suma 341
El renglón [1, 11, 55, 165, 330, 462, 462, 330, 165, 55, 11, 1] saltando de 3 en 3 tiene suma 683
El renglón [1, 12, 66, 220, 495, 792, 924, 792, 495, 220, 66, 12, 1] saltando de 3 en 3 tiene suma 1366
El renglón [1, 13, 78, 286, 715, 1287, 1716, 1716, 1287, 715, 286, 78, 13, 1] saltando de 3 en 3 tiene suma 2731
El renglón [1, 14, 91, 364, 1001, 2002, 3003, 3432, 3003, 2002, 1001, 364, 91, 14, 1] saltando de 3 en 3 tiene suma 5461
\end{verbatim}

Tampoco hay nada muy obvio sucediendo. En ocasiones algunos problemas
son así. No basta con explorar lo que nos están pidiendo, sino que
además debemos explorar otros elementos del problema que debemos
introducir por nuestra cuenta. Para este problema la clave es ver qué
sucede no sólo con las sumas que nos interesan, sino también con
aquellas cuando empezamos desfasados en
{\textbackslash(1\textbackslash)} o en {\textbackslash(2\textbackslash)}
elementos.

Así, tomemos {\textbackslash(B\_n\textbackslash)} como la suma de los
elementos del renglón {\textbackslash(n\textbackslash)} saltando de
{\textbackslash(3\textbackslash)} en {\textbackslash(3\textbackslash)},
pero comenzando en el primer elemento. La siguiente figura muestra qué
números estamos sumando.

\begin{longtable}[]{@{}llllllllllll@{}}
\toprule\noalign{}
& & & & & & & & & & & \\
\midrule\noalign{}
\endhead
\bottomrule\noalign{}
\endlastfoot
& & & & & 1 & & & & & & \\
& & & & 1 & & \textbf{1} & & & & & \\
& & & 1 & & \textbf{2} & & 1 & & & & \\
& & 1 & & \textbf{3} & & 3 & & 1 & & & \\
& 1 & & \textbf{4} & & 6 & & 4 & & \textbf{1} & & \\
1 & & \textbf{5} & & 10 & & 10 & & \textbf{5} & & 1 & \\
\end{longtable}

Y tomemos {\textbackslash(C\_n\textbackslash)} como la suma de los
elementos del renglón {\textbackslash(n\textbackslash)} saltando de
{\textbackslash(3\textbackslash)} en {\textbackslash(3\textbackslash)},
pero comenzando en el segundo elemento. La siguiente figura muestra qué
números estamos sumando.

\begin{longtable}[]{@{}llllllllllll@{}}
\toprule\noalign{}
& & & & & & & & & & & \\
\midrule\noalign{}
\endhead
\bottomrule\noalign{}
\endlastfoot
& & & & & 1 & & & & & & \\
& & & & 1 & & 1 & & & & & \\
& & & 1 & & 2 & & \textbf{1} & & & & \\
& & 1 & & 3 & & \textbf{3} & & 1 & & & \\
& 1 & & 4 & & \textbf{6} & & 4 & & 1 & & \\
1 & & 5 & & \textbf{10} & & 10 & & 5 & & \textbf{1} & \\
\end{longtable}

Ahora sí, exploremos conjuntamente a los valores de
{\textbackslash(A\_n\textbackslash)},
{\textbackslash(B\_n\textbackslash)} y
{\textbackslash(C\_n\textbackslash)}. En la siguiente tabla puedes ver
los valores para {\textbackslash(n\textbackslash)},
{\textbackslash(A\_n\textbackslash)},
{\textbackslash(B\_n\textbackslash)} y
{\textbackslash(C\_n\textbackslash)}.

\begin{longtable}[]{@{}llll@{}}
\toprule\noalign{}
Valor de {\textbackslash(n\textbackslash)} & Valor de
{\textbackslash(A\_n\textbackslash)} & Valor de
{\textbackslash(B\_n\textbackslash)} & Valor de
{\textbackslash(C\_n\textbackslash)} \\
\midrule\noalign{}
\endhead
\bottomrule\noalign{}
\endlastfoot
0 & 1 & 0 & 0 \\
1 & 1 & 1 & 0 \\
2 & 1 & 2 & 1 \\
3 & 2 & 3 & 3 \\
4 & 5 & 5 & 6 \\
5 & 11 & 10 & 11 \\
\end{longtable}

Aunque no hayamos hecho cuentas computacionales, ¡el patrón comienza a
revelarse! Parece ser que sucede todo lo siguiente:

\begin{itemize}
\item
  Para cada {\textbackslash(n\textbackslash)}, dos de los valores
  {\textbackslash(A\_n\textbackslash)},
  {\textbackslash(B\_n\textbackslash)} y
  {\textbackslash(C\_n\textbackslash)} son iguales y el tercero es
  distinto sólo en una unidad.
\item
  Alternadamente, el distinto es mayor y menor.
\item
  Ciclicamente el disinto es {\textbackslash(A\_n\textbackslash)}, luego
  {\textbackslash(C\_n\textbackslash)}, luego
  {\textbackslash(B\_n\textbackslash)}.
\item
  La suma de {\textbackslash(A\_n\textbackslash)} con
  {\textbackslash(B\_n\textbackslash)} es
  {\textbackslash(B\_\{n+1\}\textbackslash)}.
\item
  La suma de {\textbackslash(B\_n\textbackslash)} con
  {\textbackslash(C\_n\textbackslash)} es
  {\textbackslash(C\_\{n+1\}\textbackslash)}.
\item
  La suma de {\textbackslash(C\_n\textbackslash)} con
  {\textbackslash(A\_n\textbackslash)} es
  {\textbackslash(A\_\{n+1\}\textbackslash)}.
\end{itemize}

Una vez más, dejaremos pendiente la demostración de estas conjeturas
(todas ellas ciertas) y usaremos su validez de momento sin
demostración. Lo que sugieren es que para cuando
{\textbackslash(n=100\textbackslash)}, se tiene que el número distinto
a los otros dos es {\textbackslash(C\_\{100\}\textbackslash)} y que es
una unidad más grande que {\textbackslash(A\_\{100\}\textbackslash)} y
{\textbackslash(B\_\{100\}\textbackslash)}.

Pero además por el problema anterior sabemos algo crucial de
{\textbackslash(A\_\{100\}\textbackslash)},
{\textbackslash(B\_\{100\}\textbackslash)} y
{\textbackslash(C\_\{100\}\textbackslash)}: su suma es
{\textbackslash(2\^{}\{100\}\textbackslash)}. De esta manera, si
{\textbackslash(A\_\{100\}=x\textbackslash)}, entonces
{\textbackslash(B\_\{100\}=x\textbackslash)} y
{\textbackslash(C\_\{100\}=x+1\textbackslash)} y por lo tanto

\textbackslash{[}3x+1=2\^{}\{100\}.\textbackslash{]}

De aquí obtenemos el valor
{\textbackslash(A\_\{100\}=x=\textbackslash frac\{2\^{}\{100\}-1\}\{3\}\textbackslash)}
que buscábamos. {{\textbackslash(\textbackslash square\textbackslash)}}

\subsection{\texorpdfstring{Tarea
moral\hyperref[tarea-moral]{\#}}{Tarea moral\#}}\label{tarea-moral}

Los siguientes problemas te ayudarán a practicar lo visto en esta
entrada. Para resolverlos, necesitarás usar herramientas matemáticas,
computacionales o ambas.

\begin{enumerate}
\item
  Si se dibuja una recta en el plano, entonces queda dividido en dos
  regiones. Si se dibujan dos rectas (no paralelas), queda dividido en
  cuatro. Si de dibujan tres rectas (sin paralelas ni tripes
  intersecciones), queda dividido en siete. ¿Qué sucede con cuatro
  rectas? ¿Cuántas regiones nuevas se hacen? ¿Y si son en total
  {\textbackslash(10\textbackslash)} rectas? ¿Si son
  {\textbackslash(10000\textbackslash)}?
\item
  Considera la sucesión {\textbackslash(a\_n\textbackslash)} definida
  como sigue. El valor de {\textbackslash(a\_0\textbackslash)} es
  {\textbackslash(0\textbackslash)}, el de
  {\textbackslash(a\_1\textbackslash)} es
  {\textbackslash(1\textbackslash)}, el de
  {\textbackslash(a\_2\textbackslash)} es
  {\textbackslash(2\textbackslash)} y para
  {\textbackslash(n\textbackslash geq 0\textbackslash)} se cumple que
  {\textbackslash(a\_\{n+3\}=a\_\{n+2\}-a\_\{n+1\}+a\_n\textbackslash)}.
  ¿Sucederá en algún momento que {\textbackslash(a\_n\textbackslash)}
  exceda {\textbackslash(100\textbackslash)}? ¿Qué sucede si cambiamos
  los números iniciales a {\textbackslash(a\_0=3\textbackslash)},
  {\textbackslash(a\_1=5\textbackslash)} y
  {\textbackslash(a\_2=9\textbackslash)}?
\item
  Hay otra cosa interesante que sucede con el triángulo de Pascal. En
  vez de sumar por renglones, podemos sumar por \emph{diagonales}. Una
  diagonal del triángulo de Pascal consiste en comenzar con el primer
  número de un renglón, luego el segundo del renglón de arriba, luego
  el tercero del renglón de arriba y así sucesivamente hasta que ya no
  podamos sumar más.

  Un ejemplo de diagonal es el siguiente:

  \begin{longtable}[]{@{}llllllllllll@{}}
  \toprule\noalign{}
  & & & & & & & & & & & \\
  \midrule\noalign{}
  \endhead
  \bottomrule\noalign{}
  \endlastfoot
  & & & & & 1 & & & & & & \\
  & & & & 1 & & \textbf{1} & & & & & \\
  & & & \textbf{1} & & 2 & & 1 & & & & \\
  & & 1 & & 3 & & 3 & & 1 & & & \\
  & 1 & & 4 & & 6 & & 4 & & 1 & & \\
  1 & & 5 & & 10 & & 10 & & 5 & & 1 & \\
  \end{longtable}

  Si sumamos en esta diagonal, da {\textbackslash(2\textbackslash)}.

  Aquí hay otro ejemplo de una diagonal:

  \begin{longtable}[]{@{}llllllllllll@{}}
  \toprule\noalign{}
  & & & & & & & & & & & \\
  \midrule\noalign{}
  \endhead
  \bottomrule\noalign{}
  \endlastfoot
  & & & & & 1 & & & & & & \\
  & & & & 1 & & 1 & & & & & \\
  & & & 1 & & 2 & & 1 & & & & \\
  & & 1 & & 3 & & \textbf{3} & & 1 & & & \\
  & 1 & & \textbf{4} & & 6 & & 4 & & 1 & & \\
  \textbf{1} & & 5 & & 10 & & 10 & & 5 & & 1 & \\
  \end{longtable}

  Si sumamos en esta diagonal, da {\textbackslash(8\textbackslash)}.

  ¿Cuánto suman las entradas de la diagonal que comienza en el
  renglón {\textbackslash(n\textbackslash)}?
\item
  En el último problema de la entrada pudimos simplemente haberle
  pedido a Python de manera computacional que encontrara el valor
  buscado. Haz esto y verifica que coincide con la fórmula que
  encontramos. ¿Qué ventaja tiene entonces haber encontrado una
  fórmula?
\item
  ¿Será cierto que para cualquier entero positivo
  {\textbackslash(k\textbackslash)} la sucesión de Fibonacci tiene una
  infinidad de múltiplos de {\textbackslash(k\textbackslash)}?
  Modificando el código de este capítulo, realiza varios experimentos
  computacionales para conjeturar si es cierto o no.
\end{enumerate}

\href{IntroCombinatoria.html}{\emph{}}

anterior

Fundamentos de combinatoria

\href{Casillas.html}{}

siguiente

Principio de las casillas

\emph{}

\emph{} Contenido

\begin{itemize}
\tightlist
\item
  \hyperref[introduccion]{Introducción}
\item
  \hyperref[fibonaccis-y-residuos-que-dejan]{Fibonaccis y residuos que
  dejan}
\item
  \hyperref[suma-por-renglones-en-el-triangulo-de-pascal]{Suma por
  renglones en el triángulo de Pascal}
\item
  \hyperref[triangulo-de-pascal-con-saltos-de-tres-en-tres]{Triángulo
  de Pascal con saltos de tres en tres}
\item
  \hyperref[tarea-moral]{Tarea moral}
\end{itemize}

Por Leonardo Ignacio Martínez Sandoval

© Copyright 2022.\\

\end{document}
