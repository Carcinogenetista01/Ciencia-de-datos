% Options for packages loaded elsewhere
\PassOptionsToPackage{unicode}{hyperref}
\PassOptionsToPackage{hyphens}{url}
\documentclass[
]{article}
\usepackage{xcolor}
\usepackage[margin=1in]{geometry}
\usepackage{amsmath,amssymb}
\setcounter{secnumdepth}{-\maxdimen} % remove section numbering
\usepackage{iftex}
\ifPDFTeX
  \usepackage[T1]{fontenc}
  \usepackage[utf8]{inputenc}
  \usepackage{textcomp} % provide euro and other symbols
\else % if luatex or xetex
  \usepackage{unicode-math} % this also loads fontspec
  \defaultfontfeatures{Scale=MatchLowercase}
  \defaultfontfeatures[\rmfamily]{Ligatures=TeX,Scale=1}
\fi
\usepackage{lmodern}
\ifPDFTeX\else
  % xetex/luatex font selection
    \setmainfont[]{DejaVu Serif}
    \setmonofont[]{DejaVu Sans Mono}
\fi
% Use upquote if available, for straight quotes in verbatim environments
\IfFileExists{upquote.sty}{\usepackage{upquote}}{}
\IfFileExists{microtype.sty}{% use microtype if available
  \usepackage[]{microtype}
  \UseMicrotypeSet[protrusion]{basicmath} % disable protrusion for tt fonts
}{}
\makeatletter
\@ifundefined{KOMAClassName}{% if non-KOMA class
  \IfFileExists{parskip.sty}{%
    \usepackage{parskip}
  }{% else
    \setlength{\parindent}{0pt}
    \setlength{\parskip}{6pt plus 2pt minus 1pt}}
}{% if KOMA class
  \KOMAoptions{parskip=half}}
\makeatother
\ifLuaTeX
\usepackage[bidi=basic]{babel}
\else
\usepackage[bidi=default]{babel}
\fi
\babelprovide[main,import]{spanish}
\ifPDFTeX
\else
\babelfont{rm}[]{DejaVu Serif}
\fi
% get rid of language-specific shorthands (see #6817):
\let\LanguageShortHands\languageshorthands
\def\languageshorthands#1{}
\setlength{\emergencystretch}{3em} % prevent overfull lines
\providecommand{\tightlist}{%
  \setlength{\itemsep}{0pt}\setlength{\parskip}{0pt}}
\usepackage{bookmark}
\IfFileExists{xurl.sty}{\usepackage{xurl}}{} % add URL line breaks if available
\urlstyle{same}
\hypersetup{
  pdftitle={Fundamentos de combinatoria --- Matemáticas Discretas para Ciencia de Datos},
  pdflang={es},
  hidelinks,
  pdfcreator={LaTeX via pandoc}}

\title{Fundamentos de combinatoria --- Matemáticas Discretas para
Ciencia de Datos}
\author{}
\date{}


        \usepackage{fontspec}
        \usepackage{unicode-math}
        \setmainfont{DejaVu Serif}
        \setmonofont{DejaVu Sans Mono}
        \usepackage{amsmath}
        \usepackage{amssymb}
        \usepackage{graphicx}
        \usepackage{hyperref}
        \usepackage[spanish]{babel}
        
\begin{document}
\maketitle

\phantomsection\label{main-content}
{}

\emph{}

\begin{itemize}
\tightlist
\item
  \href{../_sources/P1/IntroCombinatoria.ipynb}{{ \emph{} } {.ipynb}}
\item
  { \emph{} } {.pdf}
\end{itemize}

{ \emph{} }

{}

\phantomsection\label{jb-print-docs-body}
\section{Fundamentos de combinatoria}\label{fundamentos-de-combinatoria}

\phantomsection\label{print-main-content}
\phantomsection\label{jb-print-toc}
\subsection{Contenido}\label{contenido}

\begin{itemize}
\tightlist
\item
  \hyperref[introduccion]{Introducción}
\item
  \hyperref[temario]{Temario}
\end{itemize}

\phantomsection\label{searchbox}

\phantomsection\label{fundamentos-de-combinatoria}
\section{\texorpdfstring{Fundamentos de
combinatoria\hyperref[fundamentos-de-combinatoria]{\#}}{Fundamentos de combinatoria\#}}\label{fundamentos-de-combinatoria-1}

\phantomsection\label{introduccion}
\subsection{\texorpdfstring{Introducción\hyperref[introduccion]{\#}}{Introducción\#}}\label{introducciuxe3uxb3n}

En las primeras dos partes de este texto introducimos algunos
fundamentos matemáticos de las matemáticas discretas para la ciencia
de datos. Esta primer parte consiste de repasar y profundizar en algunas
herramientas matemáticas como la inducción, el principio de las
casillas, el principio extremo, entre otras.

\subsection{\texorpdfstring{Temario\hyperref[temario]{\#}}{Temario\#}}\label{temario}

\textbf{Fundamentos combinatorios}

\begin{itemize}
\item
  Buscar un patrón
\item
  Principio de las casillas
\item
  Principio de doble conteo y coeficientes binomiales
\item
  Principio de inducción
\item
  Principio de recursión
\item
  Principio extremo
\end{itemize}

\href{../intro.html}{\emph{}}

anterior

Bienvenida

\href{BuscarPatron.html}{}

siguiente

Buscar un patrón

\emph{}

\emph{} Contenido

\begin{itemize}
\tightlist
\item
  \hyperref[introduccion]{Introducción}
\item
  \hyperref[temario]{Temario}
\end{itemize}

Por Leonardo Ignacio Martínez Sandoval

© Copyright 2022.\\

\end{document}
