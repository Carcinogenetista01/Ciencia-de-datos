\phantomsection\label{main-content}
{}

\emph{}

\begin{itemize}
\tightlist
\item
  \href{../_sources/P1/Recursion.ipynb}{{ \emph{} } {.ipynb}}
\item
  { \emph{} } {.pdf}
\end{itemize}

{ \emph{} }

{}

\phantomsection\label{jb-print-docs-body}
\section{Principio de recursión}\label{principio-de-recursiuxe3uxb3n}

\phantomsection\label{print-main-content}
\phantomsection\label{jb-print-toc}
\subsection{Contenido}\label{contenido}

\begin{itemize}
\tightlist
\item
  \hyperref[introduccion]{Introducción}
\item
  \hyperref[sucesiones-recursivas]{Sucesiones recursivas}
\item
  \hyperref[recursion-para-resolver-problemas]{Recursión para resolver
  problemas}
\item
  \hyperref[sucesiones-recursivas-lineales]{Sucesiones recursivas
  lineales}
\item
  \hyperref[tarea-moral]{Tarea moral}
\end{itemize}

\phantomsection\label{searchbox}

\phantomsection\label{principio-de-recursion}
\section{\texorpdfstring{Principio de
recursión\hyperref[principio-de-recursion]{\#}}{Principio de recursión\#}}\label{principio-de-recursiuxe3uxb3n-1}

\phantomsection\label{introduccion}
\subsection{\texorpdfstring{Introducción\hyperref[introduccion]{\#}}{Introducción\#}}\label{introducciuxe3uxb3n}

En esta entrada supondremos que conoces las nociones básicas de
sucesiones. También supondremos que conoces a grandes rasgos el teorema
de recursión en los números naturales. De no ser así, puedes revisar
el siguiente material:

\begin{itemize}
\item
  Material de sucesiones en \url{https://blog.nekomath.com/srp/}
\item
  Material de la Unidad 1 en \url{https://blog.nekomath.com/as2/}
\end{itemize}

El teorema de recursión en los números naturales nos dice, a grandes
rasgos, que podemos crear funciones en los números naturales «haciendo
referencia a términos anteriores». Esta noción la podemos llevar al
contexto de sucesiones, a la de resolución de problemas de conteo y a
la de creación de algoritmos. En esta entrada discutiremos algunas de
estas aplicaciones.

\subsection{\texorpdfstring{Sucesiones
recursivas\hyperref[sucesiones-recursivas]{\#}}{Sucesiones recursivas\#}}\label{sucesiones-recursivas}

Una sucesión en un conjunto {\textbackslash(X\textbackslash)} consiste
simplemente en definir, para cada entero
{\textbackslash(n\textbackslash geq 0\textbackslash)} un elemento
{\textbackslash(x\_n\textbackslash)} en
{\textbackslash(X\textbackslash)}. Una manera de hacer esto es
\textbf{recursivamente}, es decir, explicando cómo es el valor de un
término {\textbackslash(x\_n\textbackslash)} en función de los
términos anteriores
{\textbackslash(x\_0,x\_1,\textbackslash ldots,x\_\{n-1\}\textbackslash)}.
Si tenemos una \textbf{sucesión recursiva} así de general, basta con
definir el término {\textbackslash(x\_0\textbackslash)} y la regla
recursiva para construir toda la sucesión.

\emph{Ejemplo.} Consideremos la sucesión recursiva dada por

\textbackslash{[}\textbackslash begin\{align*\}
y\_0\&=1\textbackslash\textbackslash{}
y\_n\&=y\_0+y\_1+y\_2+\textbackslash ldots+y\_\{n-1\}+1.
\textbackslash end\{align*\}\textbackslash{]}

Su primer término es {\textbackslash(y\_0=1\textbackslash)}. Si
queremos encontrar {\textbackslash(y\_1\textbackslash)}, podemos usar la
regla recursiva en {\textbackslash(n=1\textbackslash)} para obtener

\textbackslash{[}y\_1=y\_0+1=1+1=2.\textbackslash{]}

Para encontrar su segundo término, usamos de nuevo la regla recursiva,
ahora para {\textbackslash(n=2\textbackslash)} para obtener

\textbackslash{[}y\_2=y\_0+y\_1+1=1+2+1=4.\textbackslash{]}

Calculando un par de valores más tenemos

\textbackslash{[}\textbackslash begin\{align*\}
y\_3\&=y\_0+y\_1+y\_2+1=1+2+4+1=8\textbackslash\textbackslash{}
y\_4\&=y\_0+y\_1+y\_2+y\_3+1=1+2+4+8+1=16.
\textbackslash end\{align*\}\textbackslash{]}

Si bien tenemos una fórmula recursiva, en este caso parece aparecer un
patrón que nos da una fórmula cerrada: las potencias de dos. Para
demostrar que {\textbackslash(y\_n=2\^{}n\textbackslash)} se puede
utilizar inducción fuerte.
{{\textbackslash(\textbackslash square\textbackslash)}}

En algunas ocasiones queremos que nuestra sucesión dependa de los
{\textbackslash(k\textbackslash)} últimos valores que definimos, es
decir, que {\textbackslash(x\_\{n+k+1\}\textbackslash)} esté en
términos de
{\textbackslash(x\_\{n+1\},\textbackslash ldots,x\_\{n+k\}\textbackslash)}.
Cuando este es el caso, usualmente debemos definir los primeros
{\textbackslash(k\textbackslash)} valores de la sucesión para de ahí
definir los demás. En este caso decimos que la sucesión es
\textbf{recursiva de orden} {\textbackslash(k\textbackslash)}. Un
ejemplo que ya hemos visto es la sucesión de Fibonacci: los primeros
dos términos son {\textbackslash(0\textbackslash)} y
{\textbackslash(1\textbackslash)}. Luego, cada término depende de los
dos anteriores. Así, la sucesión de Fibonacci es una sucesión
recursiva de orden {\textbackslash(2\textbackslash)}. Veamos otro
ejemplo.

\textbf{Problema.} Considera las sucesiones
{\textbackslash(a\_n\textbackslash)} y
{\textbackslash(b\_n\textbackslash)} dadas por las siguientes reglas.

\textbackslash{[}\textbackslash begin\{align*\} a\_0\&=0, a\_1=0,
a\_2=1\textbackslash\textbackslash{}
a\_\{n+3\}\&=a\_\{n\}a\_\{n+1\}a\_\{n+2\}-a\_\{n\}-a\_\{n+1\}-a\_\{n+2\}.
\textbackslash end\{align*\}\textbackslash{]}

y

\textbackslash{[}\textbackslash begin\{align*\} b\_0\&=0, b\_1=1,
b\_2=1\textbackslash\textbackslash{}
b\_\{n+3\}\&=b\_\{n\}b\_\{n+1\}b\_\{n+2\}-b\_\{n\}-b\_\{n+1\}-b\_\{n+2\}.
\textbackslash end\{align*\}\textbackslash{]}

Demuestra que una de ellas nunca excede el valor
{\textbackslash(2\textbackslash)}, mientras que la otra toma valores tan
grandes como queramos.

\emph{Solución.} Ambas son sucesiones recursivas de orden
{\textbackslash(3\textbackslash)}. Tienen exactamente la misma regla de
recurrencia. Sin embargo, sus valores iniciales difieren:
{\textbackslash(a\_1=0\textbackslash)}, mientras que
{\textbackslash(b\_1=1\textbackslash)}. Explorémos ahora estas
sucesiones de manera computacional. Esto lo hacemos en las siguientes
celdas de código, en donde mostramos los primeros
{\textbackslash(12\textbackslash)} valores de cada sucesión.

\begin{verbatim}
x,y,z=0,0,1
an=[]
for j in range(12):
    an.append(x)
    x,y,z=y,z,x*y*z-x-y-z

print(an)
\end{verbatim}

\begin{verbatim}
[0, 0, 1, -1, 0, 0, 1, -1, 0, 0, 1, -1]
\end{verbatim}

\begin{verbatim}
x,y,z=0,1,1
bn=[]
for j in range(12):
    bn.append(x)
    x,y,z=y,z,x*y*z-x-y-z

print(bn)
\end{verbatim}

\begin{verbatim}
[0, 1, 1, -2, -2, 7, 25, -380, -66152, 628510507, 15799345654000345, -656892981641024153197473821780]
\end{verbatim}

De acuerdo a la exploración computacional, parece ser que
{\textbackslash(a\_n\textbackslash)} es la que nunca excede
{\textbackslash(2\textbackslash)}. De hecho, podemos conjeturar algo
más fuerte: que {\textbackslash(a\_n\textbackslash)} es cíclica con un
ciclo de periodo {\textbackslash(4\textbackslash)} que es
{\textbackslash(0,0,1,-1\textbackslash)}. Esto puede demostrarse por
inducción.

También, parece ser que {\textbackslash(b\_n\textbackslash)} es la que
sus valores pueden ser tan grandes como queramos. Esto también puede
demostrarse por inducción, aunque la prueba no es tan directa.
{{\textbackslash(\textbackslash square\textbackslash)}}

El problema anterior nos muestra que sucesiones recursivas con la misma
regla de recursión y puntos de partida muy parecidos pueden tener
comportamientos muy diferentes. Además, la segunda sucesión debe
sugerirnos que una regla recursiva no siempre tiene una fórmula cerrada
sencilla de expresar.

\phantomsection\label{recursion-para-resolver-problemas}
\subsection{\texorpdfstring{Recursión para resolver
problemas\hyperref[recursion-para-resolver-problemas]{\#}}{Recursión para resolver problemas\#}}\label{recursiuxe3uxb3n-para-resolver-problemas}

Cuando definimos recursivamente nos basamos en «casos más pequeños
que ya hayamos hecho». Esto funciona muy bien no sólo para definir
cosas, sino también para resolver problemas. Además, tiene la ventaja
de combinarse bien con la técnica de buscar un patrón, pues para ambas
resulta de mucha utilidad resolver casos pequeños.

Hay algunos problemas de conteo que se prestan para realizar soluciones
recursivas. Veamos un ejemplo:

\textbf{Problema.} ¿Cuántas palabras de
{\textbackslash(100\textbackslash)} letras
{\textbackslash(A\textbackslash)}, {\textbackslash(B\textbackslash)} o
{\textbackslash(C\textbackslash)} hay, en las cuales nunca aparece
{\textbackslash(AB\textbackslash)} en la palabra?

\emph{Solución.} Para una letra hay tres posibles palabras:
{\textbackslash(A\textbackslash)}, {\textbackslash(B\textbackslash)} y
{\textbackslash(C\textbackslash)}. Para dos letras hay 8 posibles
palabras: {\textbackslash(AA\textbackslash)},
{\textbackslash(AC\textbackslash)}, {\textbackslash(BA\textbackslash)},
{\textbackslash(BB\textbackslash)}, {\textbackslash(BC\textbackslash)},
{\textbackslash(CA\textbackslash)}, {\textbackslash(CB\textbackslash)} y
{\textbackslash(CC\textbackslash)}. En cuanto tenemos tres letras los
casos se empiezan a complicar. Para resolver el problema hay que
introducir dos ideas clave: partir nuestra cuentra en tres partes y
establecer una recursión.

Así, sean {\textbackslash(a\_n\textbackslash)},
{\textbackslash(b\_n\textbackslash)} y
{\textbackslash(c\_n\textbackslash)} las palabras con
{\textbackslash(n\textbackslash)} letras que nos interesan y que,
respectivamente, terminan en {\textbackslash(A\textbackslash)},
{\textbackslash(B\textbackslash)} o {\textbackslash(C\textbackslash)}.
Por los primeros casos que hicimos tenemos:

\textbackslash{[}\textbackslash begin\{align*\} a\_1=1, b\_1=1,
c\_1=1\textbackslash\textbackslash{} a\_2=3, b\_2=2, c\_2=3.
\textbackslash end\{align*\}\textbackslash{]}

La ventaja de partir el problema y estudiarlo más en general es que
ahora podemos traducir la hipótesis de que nunca aparece
{\textbackslash(AB\textbackslash)} en una condición algebraica
recursiva. Tenemos que:

\textbackslash{[}\textbackslash begin\{align*\}
a\_\{n+1\}\&=a\_n+b\_n+c\_n\textbackslash\textbackslash{}
b\_\{n+1\}\&=b\_n+c\_n\textbackslash\textbackslash{}
c\_\{n+1\}\&=a\_n+b\_n+c\_n.
\textbackslash end\{align*\}\textbackslash{]}

La primera igualdad se debe a que una palabra de
{\textbackslash(n+1\textbackslash)} letras que nos interesa y que
termine en {\textbackslash(A\textbackslash)} puede ser formada a partir
de agregar {\textbackslash(A\textbackslash)} al final a una de
{\textbackslash(n\textbackslash)} letras que termine en
{\textbackslash(a\textbackslash)}, o una de
{\textbackslash(n\textbackslash)} letras que termine en
{\textbackslash(B\textbackslash)}, o una de
{\textbackslash(n\textbackslash)} letras que termine en
{\textbackslash(C\textbackslash)}. La explicación es análoga para la
tercera igualdad. Sin embargo, si una palabra de
{\textbackslash(n+1\textbackslash)} letras de las que nos interesan
termina en {\textbackslash(B\textbackslash)}, no pudo haber sido al
agregarle una {\textbackslash(B\textbackslash)} al final a una que
terminara en {\textbackslash(A\textbackslash)}, pues en ese caso se
crearía el patrón {\textbackslash(AB\textbackslash)} prohibido.

Ya argumentada la validez de la recursión, podemos encontrar
computacionalmente los valores
{\textbackslash(a\_\{100\},b\_\{100\},c\_\{100\}\textbackslash)} que nos
interesan para sumarlos y dar el resultado pedido. Hacemos esto a
continuación.

\begin{verbatim}
a,b,c=1,1,1
for j in range(99):
    a,b,c=a+b+c,b+c,a+b+c
print(a+b+c)
\end{verbatim}

\begin{verbatim}
734544867157818093234908902110449296423351
\end{verbatim}

{{\textbackslash(\textbackslash square\textbackslash)}}

Veamos ahora un problema que viene de datos reales.

\textbf{Problema.} En el siguiente código se carga la información de
la inflación por año en México desde 1970 a 2020.

\begin{verbatim}
import pandas as pd

df_inflacion=pd.read_csv('Inflacion.csv',index_col="Year")
display(df_inflacion.head(3))
print("...")
display(df_inflacion.tail(2))
\end{verbatim}

\begin{longtable}[]{@{}ll@{}}
\toprule\noalign{}
& Inflation \\
Year & \\
\midrule\noalign{}
\endhead
\bottomrule\noalign{}
\endlastfoot
1970 & 4.69 \\
1971 & 4.96 \\
1972 & 5.56 \\
\end{longtable}

\begin{verbatim}
...
\end{verbatim}

\begin{longtable}[]{@{}ll@{}}
\toprule\noalign{}
& Inflation \\
Year & \\
\midrule\noalign{}
\endhead
\bottomrule\noalign{}
\endlastfoot
2019 & 2.83 \\
2020 & 3.15 \\
\end{longtable}

Se quiere saber cuál es la máxima cantidad de años que se pueden
tomar, de menor a mayor, pero no necesariamente seguidos, para los
cuales la inflación disminuyó. Por ejemplo, se puede verificar que
{\textbackslash(1980, 1989, 1999, 2004, 2009\textbackslash)} nos da
cinco años en los cuales la inflación disminuyó, pues tuvo los
siguientes valores:

\begin{verbatim}
df_inflacion.loc[[1980,1989,1999,2004,2009]]
\end{verbatim}

\begin{longtable}[]{@{}ll@{}}
\toprule\noalign{}
& Inflation \\
Year & \\
\midrule\noalign{}
\endhead
\bottomrule\noalign{}
\endlastfoot
1980 & 29.84 \\
1989 & 19.69 \\
1999 & 12.32 \\
2004 & 5.19 \\
2009 & 3.57 \\
\end{longtable}

¿Será esta la lista más larga de años que podemos conseguir? La idea
para resolver el problema es pensar de manera recursiva. La lista más
larga que termina en {\textbackslash(1970\textbackslash)} consiste de
únicamente ese año. ¿Cuál es la lista más larga que termina en el
año {\textbackslash(n\textbackslash)}?

\begin{itemize}
\item
  Si todas las inflaciones anteriores son más chicas, entonces la lista
  consistirá sólo del año {\textbackslash(n\textbackslash)}.
\item
  Si en algún año {\textbackslash(k\textbackslash)} anterior la
  inflación es mayor, entonces entonces el año
  {\textbackslash(n\textbackslash)} puede alargar la lista más larga
  que termina en {\textbackslash(k\textbackslash)}.
\end{itemize}

Así, la lista más larga que termina en el año
{\textbackslash(n\textbackslash)} será de
{\textbackslash(1\textbackslash)} año, o de lo máximo que podamos
extender una que termine en una inflación menor.

Estas ideas nos llevan a la siguiente solución en código. De momento
no discutiremos los detalles y los veremos con un poco más de
profundidad en la parte 4 del libro cuando hablemos de heurísticas para
la creación de algoritmos.

\begin{verbatim}
lista=df_inflacion['Inflation'].to_list()
max_decreciente=[]
padres=[]

for j in range(51):
    mejor=0
    padre=None
    for k in range(j-1):
        if lista[k]>lista[j] and max_decreciente[k]>mejor:
            padre=k
            mejor=max_decreciente[k]
    padres.append(padre)
    max_decreciente.append(mejor+1)

mejor_lista=[]
padre=49
while padre!=None:
    mejor_lista=[padre]+mejor_lista
    padre=padres[mejor_lista[0]]
mejor_lista=[x+1970 for x in mejor_lista]
print(mejor_lista)

df_inflacion.loc[mejor_lista]
\end{verbatim}

\begin{verbatim}
[1973, 1977, 1979, 1989, 1991, 1997, 1999, 2002, 2004, 2006, 2009, 2016, 2019]
\end{verbatim}

\begin{longtable}[]{@{}ll@{}}
\toprule\noalign{}
& Inflation \\
Year & \\
\midrule\noalign{}
\endhead
\bottomrule\noalign{}
\endlastfoot
1973 & 21.37 \\
1977 & 20.66 \\
1979 & 20.02 \\
1989 & 19.69 \\
1991 & 18.79 \\
1997 & 15.72 \\
1999 & 12.32 \\
2002 & 5.70 \\
2004 & 5.19 \\
2006 & 4.05 \\
2009 & 3.57 \\
2016 & 3.36 \\
2019 & 2.83 \\
\end{longtable}

De este modo, la mejor lista de años en los cuales la inflación ha
disminuido tiene longitud {\textbackslash(13\textbackslash)}.
{{\textbackslash(\textbackslash square\textbackslash)}}

\subsection{\texorpdfstring{Sucesiones recursivas
lineales\hyperref[sucesiones-recursivas-lineales]{\#}}{Sucesiones recursivas lineales\#}}\label{sucesiones-recursivas-lineales}

Las sucesiones recursivas pueden ser tan extrañas como uno quisiera.
Por ejemplo, podríamos pedir que {\textbackslash(a\_0=1, a\_1=2,
a\_2=3\textbackslash)} y que
{\textbackslash(a\_n=a\_\{n-1\}+a\_\{n-2\}\^{}2-\textbackslash log
a\_\{n-3\}\textbackslash)}. En general es difícil entender totalmente
una sucesión recursiva arbitraria y este tipo de preguntas nos llevan a
áreas muy interesantes de las matemáticas como la dinámica y los
sistemas caóticos. Por esta razón, es de mucha utilidad clasificar a
las recursiones en distintos tipos e identificar aquellos que podamos
estudiar de manera sencilla.

\textbf{Definición.} Para un entero {\textbackslash(k\textbackslash)}
decimos que una sucesión es \textbf{recursiva lineal de orden}
{\textbackslash(k\textbackslash)} si:

\begin{itemize}
\item
  Se dan sus primeros {\textbackslash(k\textbackslash)} términos
  {\textbackslash(a\_0,\textbackslash ldots,a\_\{k-1\}\textbackslash)} y
\item
  Satisface una \textbf{recursión lineal}, es decir existen
  {\textbackslash(k\textbackslash)} constantes
  {\textbackslash(c\_0,\textbackslash ldots,c\_\{k-1\}\textbackslash)}
  tales que para todo {\textbackslash(n\textbackslash geq
  0\textbackslash)} se cumple que

  \textbackslash{[}a\_\{n+k\}=c\_0a\_\{n\}+c\_1a\_\{n+1\}+\textbackslash ldots+c\_\{k-1\}a\_\{n+k-1\}.\textbackslash{]}
\end{itemize}

En términos de álgebra lineal, debe existir una
{\textbackslash(k\textbackslash)} fija y
{\textbackslash(k\textbackslash)} coeficientes fijos tales que cada
término es combinación lineal de los {\textbackslash(k\textbackslash)}
anteriores con esos coeficientes.

\emph{Ejemplo.} Las suceciones recursivas lineales de orden
{\textbackslash(1\textbackslash)} son simplemente las sucesiones
geométricas pues se nos da el primer término
{\textbackslash(a\_0\textbackslash)} y luego la recursión que se cumple
es {\textbackslash(a\_\{n+1\}=c\_0a\_n\textbackslash)}. Una sencilla
prueba inductiva muestra que

\textbackslash{[}a\_n=a\_0\textbackslash cdot
c\_0\^{}n.\textbackslash{]}

Así, a partir de una fórmula recursiva es muy sencillo dar una fórmula
cerrada para ellas.
{{\textbackslash(\textbackslash square\textbackslash)}}

\emph{Ejemplo.} Ya nos hemos encontrado con una sucesión recursiva
lineal de orden {\textbackslash(2\textbackslash)}: la sucesión de
Fibonacci. Se dan los primeros dos términos como
{\textbackslash(0\textbackslash)} y {\textbackslash(1\textbackslash)}.
Luego, cada término es la suma de los dos anteriores, es decir es la
combinación lineal de ellos con coeficiente
{\textbackslash(1\textbackslash)}.

Otra sucesión recursiva lineal de orden
{\textbackslash(2\textbackslash)} es la sucesión de Lucas. Los primeros
dos términos son {\textbackslash(2\textbackslash)} y
{\textbackslash(1\textbackslash)}. Luego, cada término es la suma de
los dos anteriores. Los primeros números de la sucesión de Lucas son:

\textbackslash{[}2,1,3,4,7,11,18,29,47,76,\textbackslash ldots.\textbackslash{]}

{{\textbackslash(\textbackslash square\textbackslash)}}

Las sucesiones recursivas lineales se conocen muy bien. Un primer
resultado del cual no es difícil convencerse mediante un argumento
inductivo es el siguiente.

\textbf{Lema.} Si dos sucesiones recursivas lineales de orden
{\textbackslash(k\textbackslash)} coinciden en sus primeros
{\textbackslash(k\textbackslash)} elementos y además tienen la misma
regla recursiva, entonces coinciden en todos sus elementos.

A veces esto es suficiente para resolver un problema.

\textbf{Problema.} Demuesta la siguiente fórmula para los números de
Fibonacci:

\textbackslash{[}F\_n=\textbackslash frac\{\textbackslash left(\textbackslash frac\{1+\textbackslash sqrt\{5\}\}\{2\}\textbackslash right)\^{}n-\textbackslash left(\textbackslash frac\{1-\textbackslash sqrt\{5\}\}\{2\}\textbackslash right)\^{}n\}\{\textbackslash sqrt\{5\}\}.\textbackslash{]}

\emph{Solución.} Llamemos {\textbackslash(f\_n\textbackslash)} al lado
derecho de la fórmula por demostrar. Una verificación rápida muestra
{\textbackslash(f\_0=0=F\_0\textbackslash)} y que
{\textbackslash(f\_1=1=F\_1\textbackslash)}. Por el lema anterior,
bastaría ver que {\textbackslash(f\_n\textbackslash)} es recursiva
lineal de orden {\textbackslash(2\textbackslash)}, con la misma regla de
recursión que la sucesión de Fibonacci.

Escribamos
{\textbackslash(\textbackslash varphi=\textbackslash frac\{1+\textbackslash sqrt\{5\}\}\{2\}\textbackslash)}
y
{\textbackslash(\textbackslash varphi\textquotesingle=\textbackslash frac\{1-\textbackslash sqrt\{5\}\}\{2\}\textbackslash)}.
Haciendo las operaciones, se puede verificar que
{\textbackslash((x-\textbackslash varphi)(x-\textbackslash varphi\textquotesingle)=x\^{}2-x-1\textbackslash)},
de modo que {\textbackslash(\textbackslash varphi\textbackslash)} y
{\textbackslash(\textbackslash varphi\textquotesingle\textbackslash)}
son las raíces del polinomio de la derecha. Esto nos dice que
{\textbackslash(\textbackslash varphi\^{}2=\textbackslash varphi+1\textbackslash)}
y que
{\textbackslash(\textbackslash varphi\textquotesingle\^{}2=\textbackslash varphi\textquotesingle+1\textbackslash)}.
Al multiplicar por
{\textbackslash(\textbackslash varphi\^{}n\textbackslash)} y
{\textbackslash(\textbackslash varphi\textquotesingle\^{}n\textbackslash)},
respectivamente, obtenemos las igualdades:

\textbackslash{[}\textbackslash begin\{align*\}
\textbackslash varphi\^{}\{n+2\}\&=\textbackslash varphi\^{}\{n+1\}+\textbackslash varphi\^{}n,\textbackslash\textbackslash{}
\textbackslash varphi\textquotesingle\^{}\{n+2\}\&=\textbackslash varphi\textquotesingle\^{}\{n+1\}+\textbackslash varphi\textquotesingle\^{}n.\textbackslash\textbackslash{}
\textbackslash end\{align*\}\textbackslash{]}

Restando la segunda a la primera y dividiendo entre
{\textbackslash(\textbackslash sqrt\{5\}\textbackslash)} obtenemos que
{\textbackslash(f\_\{n+2\}=f\_\{n+1\}+f\_n\textbackslash)}. Así, la
sucesión {\textbackslash(F\_n\textbackslash)} y la
{\textbackslash(f\_n\textbackslash)} tienen los mismos dos términos y
la misma recursión lineal de orden {\textbackslash(2\textbackslash)}.
De este modo, por el lema deben coincidir.
{{\textbackslash(\textbackslash square\textbackslash)}}

El problema anterior es lindo, pero no explica de dónde sale la
fórmula. La clave está en el polinomio
{\textbackslash(x\^{}2-x-1\textbackslash)} que se menciona durante el
problema. Si tenemos {\textbackslash(a\_n\textbackslash)} una sucesión
recursiva lineal de orden {\textbackslash(2\textbackslash)} con
recursión
{\textbackslash(a\_\{n+2\}=c\_0a\_n+c\_1a\_\{n+1\}\textbackslash)},
entonces podemos asociarle el \textbf{polinomio característico}
{\textbackslash(x\^{}2-c\_1x-c\_0\textbackslash)}. Este polinomio está
muy conectado con la sucesión. Por ejemplo, puede mostrarse que si
tiene dos raíces distintas {\textbackslash(r\_1\textbackslash)} y
{\textbackslash(r\_2\textbackslash)}, entonces existen constantes
{\textbackslash(s\_1\textbackslash)} y
{\textbackslash(s\_2\textbackslash)} tales que
{\textbackslash(a\_n=s\_1r\_1\^{}n+s\_2r\_2\^{}n\textbackslash)}. Estas
constantes pueden encontrarse con un sistema de ecuaciones usando los
primeros valores de la sucesión. Veamos cómo podemos poner esto en
acción.

\textbf{Problema.} Encuentra una fórmula cerrada para la sucesión
{\textbackslash(x\_n\textbackslash)} tal que
{\textbackslash(x\_0=3\textbackslash)},
{\textbackslash(x\_1=5\textbackslash)} y para
{\textbackslash(n\textbackslash geq 0\textbackslash)} cumple que

\textbackslash{[}x\_\{n+2\}=2x\_\{n+1\}+3x\_n.\textbackslash{]}

\emph{Solución.} Por lo discutido arriba, debemos estudiar el polinomio
{\textbackslash(x\^{}2-2x-3\textbackslash)}. Sus raíces son
{\textbackslash(3\textbackslash)} y {\textbackslash(-1\textbackslash)},
que son distintas. Así, deben existir constantes
{\textbackslash(a\textbackslash)} y {\textbackslash(b\textbackslash)}
tales que {\textbackslash(x\_n=a\textbackslash cdot 3\^{}n +
b\textbackslash cdot (-1)\^{}n\textbackslash)}. Los casos
{\textbackslash(n=0\textbackslash)} y
{\textbackslash(n=1\textbackslash)} nos llevan al sistema de ecuaciones:

\textbackslash{[}\textbackslash begin\{split\}\textbackslash begin\{cases\}a+b\&=3\textbackslash\textbackslash3a-b\&=5.
\textbackslash end\{cases\}\textbackslash end\{split\}\textbackslash{]}

La solución es {\textbackslash(a=2\textbackslash)},
{\textbackslash(b=1\textbackslash)}. De este modo, concluimos que una
fórmula cerrada para la sucesión es

\textbackslash{[}x\_n=2\textbackslash cdot 3\^{}n +
(-1)\^{}n.\textbackslash{]}

{{\textbackslash(\textbackslash square\textbackslash)}}

Hay que reflexionar un poco para convencerse de que el método anterior
es válido matemáticamente. Sin embargo, por el momento podemos
calcular computacionalmente los primeros valores de la sucesión y de la
fórmula para ir convenciéndonos de su validez.

\begin{verbatim}
a,b=3,5
valores=[]
for j in range(10):
    valores.append(a)
    a,b=b,2*b+3*a
print("Los primeros 10 valores usando la recursión son {}".format(valores))

valores=[]
for j in range(10):
    valores.append(2*3**j+(-1)**j)
print("Los primeros 10 valores usando la fórmula son {}".format(valores))
\end{verbatim}

\begin{verbatim}
Los primeros 10 valores usando la recursión son [3, 5, 19, 53, 163, 485, 1459, 4373, 13123, 39365]
Los primeros 10 valores usando la fórmula son [3, 5, 19, 53, 163, 485, 1459, 4373, 13123, 39365]
\end{verbatim}

\subsection{\texorpdfstring{Tarea
moral\hyperref[tarea-moral]{\#}}{Tarea moral\#}}\label{tarea-moral}

Los siguientes problemas te ayudarán a practicar lo visto en esta
entrada. Para resolverlos, necesitarás usar herramientas matemáticas,
computacionales o ambas.

\begin{enumerate}
\item
  Se tienen 5 piratas, que van a repartir un tesoro de 1000 monedas
  entre ellos. Están ordenados por rangos. Repartirán el tesoro
  haciendo varias rondas de propuesta-votación. Cada una de ellas
  consiste de lo siguiente:

  \begin{itemize}
  \item
    De entre los piratas que queden, el de mayor rango hace una
    propuesta de cómo repartir las monedas.
  \item
    Se vota la propuesta entre los piratas.
  \item
    Si los votos por aprobarla son estrictamente mayor a la mitad de los
    piratas restantes, entonces así se hace la repartición.
  \item
    Si la propuesta no se aprueba con más de la mitad de los votos,
    entonces tiran al pirata que la propuso por la borda.
  \end{itemize}

  La prioridad número uno de un pirata es sobrevivir. La prioridad
  número dos es conseguir la mayor cantidad posible de monedas.
  Finalmente, si debe decidir entre dos opciones en sobrevive con la
  misma cantidad de monedas, entonces elegirá en la que más piratas se
  avienten por la borda. Los piratas conocen estas reglas y por el honor
  pirata las seguirán al pie de la letra.

  ¿Qué es lo mejor que le puede pasar a cada pirata? ¿Cómo debe
  votar? En caso de que le toque proponer, ¿cómo debe proponer? Como
  sugerencia, estudia primero qué pasa con pocos piratas.
\item
  Hay algunos problemas que quedaron pendientes en el texto. Están
  recopilados a continuación. Resuélvelos.

  \begin{itemize}
  \item
    Ver que en efecto {\textbackslash(y\_n=2\^{}n\textbackslash)}.
  \item
    Ver que la sucesión de {\textbackslash(a\_n\textbackslash)} es
    cíclica.
  \item
    Ver que la sucesión de {\textbackslash(b\_n\textbackslash)} puede
    ser tan grande como se quiere. Como sugerencia, pon a
    {\textbackslash(b\_\{n+4\}\textbackslash)} en términos de
    {\textbackslash(b\_\{n\},b\_\{n+1\},b\_\{n+2\}\textbackslash)} para
    mostrar que
    {\textbackslash(\textbar b\_\{n+4\}\textbar\textbackslash geq
    2\textbar b\_n\textbar\textbackslash)}.
  \end{itemize}
\item
  En el problema de la cuenta de palabras que no tenían
  {\textbackslash(AB\textbackslash)} encontramos el valor que nos
  pidieron. Sin embargo, todo lo obtenido permite encontrar una fórmula
  cerrada para {\textbackslash(n\textbackslash)} letras. Encuéntrala y
  demuéstrala.
\item
  Adapta la solución de la lista de años de inflación decreciente
  para ahora mostrar la lista más larga de años con solución
  creciente.
\item
  Lee la siguiente entrada de blog para conocer más de las sucesiones
  recursivas y de las recursiones lineales:
  \href{https://blog.nekomath.com/seminario-de-resolucion-de-problemas-sucesiones-recursivas-y-recursiones-lineales/}{Sucesiones
  recursivas y recursiones lineales} .
\end{enumerate}

\href{Induccion.html}{\emph{}}

anterior

Principio de inducción

\href{Extremo.html}{}

siguiente

Principio extremo

\emph{}

\emph{} Contenido

\begin{itemize}
\tightlist
\item
  \hyperref[introduccion]{Introducción}
\item
  \hyperref[sucesiones-recursivas]{Sucesiones recursivas}
\item
  \hyperref[recursion-para-resolver-problemas]{Recursión para resolver
  problemas}
\item
  \hyperref[sucesiones-recursivas-lineales]{Sucesiones recursivas
  lineales}
\item
  \hyperref[tarea-moral]{Tarea moral}
\end{itemize}

Por Leonardo Ignacio Martínez Sandoval

© Copyright 2022.\\
