% Options for packages loaded elsewhere
\PassOptionsToPackage{unicode}{hyperref}
\PassOptionsToPackage{hyphens}{url}
\documentclass[
]{article}
\usepackage{xcolor}
\usepackage[margin=1in]{geometry}
\usepackage{amsmath,amssymb}
\setcounter{secnumdepth}{-\maxdimen} % remove section numbering
\usepackage{iftex}
\ifPDFTeX
  \usepackage[T1]{fontenc}
  \usepackage[utf8]{inputenc}
  \usepackage{textcomp} % provide euro and other symbols
\else % if luatex or xetex
  \usepackage{unicode-math} % this also loads fontspec
  \defaultfontfeatures{Scale=MatchLowercase}
  \defaultfontfeatures[\rmfamily]{Ligatures=TeX,Scale=1}
\fi
\usepackage{lmodern}
\ifPDFTeX\else
  % xetex/luatex font selection
    \setmainfont[]{DejaVu Serif}
    \setmonofont[]{DejaVu Sans Mono}
\fi
% Use upquote if available, for straight quotes in verbatim environments
\IfFileExists{upquote.sty}{\usepackage{upquote}}{}
\IfFileExists{microtype.sty}{% use microtype if available
  \usepackage[]{microtype}
  \UseMicrotypeSet[protrusion]{basicmath} % disable protrusion for tt fonts
}{}
\makeatletter
\@ifundefined{KOMAClassName}{% if non-KOMA class
  \IfFileExists{parskip.sty}{%
    \usepackage{parskip}
  }{% else
    \setlength{\parindent}{0pt}
    \setlength{\parskip}{6pt plus 2pt minus 1pt}}
}{% if KOMA class
  \KOMAoptions{parskip=half}}
\makeatother
\usepackage{graphicx}
\makeatletter
\newsavebox\pandoc@box
\newcommand*\pandocbounded[1]{% scales image to fit in text height/width
  \sbox\pandoc@box{#1}%
  \Gscale@div\@tempa{\textheight}{\dimexpr\ht\pandoc@box+\dp\pandoc@box\relax}%
  \Gscale@div\@tempb{\linewidth}{\wd\pandoc@box}%
  \ifdim\@tempb\p@<\@tempa\p@\let\@tempa\@tempb\fi% select the smaller of both
  \ifdim\@tempa\p@<\p@\scalebox{\@tempa}{\usebox\pandoc@box}%
  \else\usebox{\pandoc@box}%
  \fi%
}
% Set default figure placement to htbp
\def\fps@figure{htbp}
\makeatother
\usepackage{svg}
\ifLuaTeX
\usepackage[bidi=basic]{babel}
\else
\usepackage[bidi=default]{babel}
\fi
\babelprovide[main,import]{spanish}
\ifPDFTeX
\else
\babelfont{rm}[]{DejaVu Serif}
\fi
% get rid of language-specific shorthands (see #6817):
\let\LanguageShortHands\languageshorthands
\def\languageshorthands#1{}
\setlength{\emergencystretch}{3em} % prevent overfull lines
\providecommand{\tightlist}{%
  \setlength{\itemsep}{0pt}\setlength{\parskip}{0pt}}
\usepackage{bookmark}
\IfFileExists{xurl.sty}{\usepackage{xurl}}{} % add URL line breaks if available
\urlstyle{same}
\hypersetup{
  pdftitle={Principio de las casillas --- Matemáticas Discretas para Ciencia de Datos},
  pdflang={es},
  hidelinks,
  pdfcreator={LaTeX via pandoc}}

\title{Principio de las casillas --- Matemáticas Discretas para Ciencia
de Datos}
\author{}
\date{}


        \usepackage{fontspec}
        \usepackage{unicode-math}
        \setmainfont{DejaVu Serif}
        \setmonofont{DejaVu Sans Mono}
        \usepackage{amsmath}
        \usepackage{amssymb}
        \usepackage{graphicx}
        \usepackage{hyperref}
        \usepackage[spanish]{babel}
        
\begin{document}
\maketitle

\phantomsection\label{main-content}
{}

\emph{}

\begin{itemize}
\tightlist
\item
  \href{../_sources/P1/Casillas.ipynb}{{ \emph{} } {.ipynb}}
\item
  { \emph{} } {.pdf}
\end{itemize}

{ \emph{} }

{}

\phantomsection\label{jb-print-docs-body}
\section{Principio de las casillas}\label{principio-de-las-casillas}

\phantomsection\label{print-main-content}
\phantomsection\label{jb-print-toc}
\subsection{Contenido}\label{contenido}

\begin{itemize}
\tightlist
\item
  \hyperref[introduccion]{Introducción}
\item
  \hyperref[principio-de-las-casillas-version-basica]{Principio de las
  casillas, versión básica}
\item
  \hyperref[principio-de-las-casillas-version-con-mas-objetos]{Principio
  de las casillas, versión con más objetos}
\item
  \hyperref[principio-de-las-casillas-version-infinita]{Principio de las
  casillas, versión infinita}
\item
  \hyperref[tarea-moral]{Tarea moral}
\end{itemize}

\phantomsection\label{searchbox}

\phantomsection\label{principio-de-las-casillas}
\section{\texorpdfstring{Principio de las
casillas\hyperref[principio-de-las-casillas]{\#}}{Principio de las casillas\#}}\label{principio-de-las-casillas-1}

\phantomsection\label{introduccion}
\subsection{\texorpdfstring{Introducción\hyperref[introduccion]{\#}}{Introducción\#}}\label{introducciuxe3uxb3n}

El principio de las casillas es una idea sencilla. A grandes rasgos, nos
dice que si hay muchos elementos que se deben repartir en pocas
casillas, entonces debe haber una casilla con muchos objetos. Un ejemplo
muy sencillo es que si tenemos 13 personas, entonces debe haber por lo
menos dos que compartan el mismo mes de cumpleaños. Aunque la idea de
principio de las casillas sea simple, se presta a muchas aplicaciones y
generalizaciones.

\phantomsection\label{principio-de-las-casillas-version-basica}
\subsection{\texorpdfstring{Principio de las casillas, versión
básica\hyperref[principio-de-las-casillas-version-basica]{\#}}{Principio de las casillas, versión básica\#}}\label{principio-de-las-casillas-versiuxe3uxb3n-buxe3sica}

El principio de las casillas, en su versión más básica, dice lo
siguiente:

\textbf{Proposición.} Si se tienen {\textbackslash(n+1\textbackslash)}
objetos que se deben repartir en {\textbackslash(n\textbackslash)}
casillas, entonces debe haber al menos una casilla en donde se colocaron
al menos dos objetos.

Veamos algunos problemas en donde podemos aplicar este principio para
resolverlos.

\textbf{Problema.} Encuentra la máxima cantidad de reyes de ajedrez que
puedes poner en un tablero de {\textbackslash(5\textbackslash times
5\textbackslash)} sin que se ataquen entre sí.

\emph{Solución.} Hay una forma sencilla de poner
{\textbackslash(9\textbackslash)} reyes sin que se ataquen entre sí.

\pandocbounded{\includesvg[keepaspectratio]{../_images/reyes-acomodo.svg}}

¿Cómo podemos estar totalmente seguros de que no hay ninguna forma de
poner {\textbackslash(10\textbackslash)} o más reyes? Dividamos el
tablero como sigue:

\pandocbounded{\includesvg[keepaspectratio]{../_images/reyes-casillas.svg}}

Aquí hemos creado {\textbackslash(9\textbackslash)} regiones. Si
pusiéramos {\textbackslash(10\textbackslash)} reyes, entonces habría
dos que caen en la misma región. Dos reyes en una misma región
claramente se atacarían entre sí. Así, es imposible poner
{\textbackslash(10\textbackslash)} reyes o más, y por lo tanto lo
máximo que podemos poner son {\textbackslash(9\textbackslash)}.
{{\textbackslash(\textbackslash square\textbackslash)}}

En el problema anterior las casillas son las regiones que creamos. Las
casillas no necesariamente son algo geométrico, y pueden referirse a
ideas más abstractas.

\textbf{Problema.} Muestra que para cualquier entero positivo
{\textbackslash(n\textbackslash)}, existe un múltiplo de
{\textbackslash(n\textbackslash)} que consiste de puros dígidos cero y
uno.

\emph{Solución.} Consideremos el conjunto de
{\textbackslash(n+1\textbackslash)} números
{\textbackslash(S=\textbackslash\{1,11,111,\textbackslash ldots,11\textbackslash cdots
1\textbackslash\}\textbackslash)}, en donde el último número tiene
{\textbackslash(n+1\textbackslash)} unos. Cuando dividimos cualquier
entero entre {\textbackslash(n\textbackslash)}, sólo hay
{\textbackslash(n\textbackslash)} posibilidades para el residuo que
pueden dejar:
{\textbackslash(0,1,\textbackslash ldots,n-1\textbackslash)}. Como hay
{\textbackslash(n+1\textbackslash)} números en
{\textbackslash(S\textbackslash)} y {\textbackslash(n\textbackslash)}
posibles residuos, por el principio de las casillas hay dos elementos en
{\textbackslash(S\textbackslash)} que dejan el mismo residuo al
dividirlos entre {\textbackslash(n\textbackslash)}, digamos
{\textbackslash(a\textbackslash)} y {\textbackslash(b\textbackslash)}
con {\textbackslash(a\textbackslash lt b\textbackslash)}. De esta
manera, {\textbackslash(b-a\textbackslash)} es un número que es
múltiplo de {\textbackslash(n\textbackslash)} y que consiste
únicamente de ceros y unos.
{{\textbackslash(\textbackslash square\textbackslash)}}

\phantomsection\label{principio-de-las-casillas-version-con-mas-objetos}
\subsection{\texorpdfstring{Principio de las casillas, versión con más
objetos\hyperref[principio-de-las-casillas-version-con-mas-objetos]{\#}}{Principio de las casillas, versión con más objetos\#}}\label{principio-de-las-casillas-versiuxe3uxb3n-con-muxe3s-objetos}

Si aumentamos la cantidad de objetos que debemos repartir, pero dejamos
fija la cantidad de casillas, entonces es posible garantizar todavía
más: que hay una casilla con muchos objetos. Por ejemplo, si tenemos
{\textbackslash(25\textbackslash)} personas, ahora podemos garantizar
que hay {\textbackslash(3\textbackslash)} que tienen el mismo mes de
cumpleaños. La versión precisa está dada por el siguiente resultado:

\textbf{Proposición.} Si se tienen {\textbackslash(kn+1\textbackslash)}
objetos o más que se deben repartir en
{\textbackslash(n\textbackslash)} casillas, entonces debe haber al menos
una casilla en donde se colocaron al menos
{\textbackslash(k+1\textbackslash)} objetos.

La justificación es sencilla argumentando al revés: si en cada casilla
hubiera a lo mucho {\textbackslash(k\textbackslash)} objetos, entonces
en total tendríamos a lo mucho {\textbackslash(kn\textbackslash)}
objetos.

\textbf{Problema.} Demuestra que si se colocan
{\textbackslash(9\textbackslash)} puntos dentro de un cuadrado de lado
{\textbackslash(4\textbackslash)}, entonces habrá tres de ellos que
hagan un triángulo con área menor o igual a
{\textbackslash(2\textbackslash)}.

\emph{Solución.} Dividamos al cuadrado en
{\textbackslash(4\textbackslash)} cuadraditos de lado
{\textbackslash(2\textbackslash)}. Como tenemos
{\textbackslash(9=2\textbackslash cdot 4 +1\textbackslash)} puntos,
entonces hay uno de estos cuadraditos que tiene a tres de los puntos que
colocamos.

\pandocbounded{\includesvg[keepaspectratio]{../_images/cuadrado22.svg}}

Afirmamos que el área de un triángulo
{\textbackslash(ABC\textbackslash)} dentro de un cuadrado de lado
{\textbackslash(2\textbackslash)} es como mucho
{\textbackslash(2\textbackslash)}.

Lo primero que observamos es que podemos suponer que los vértices
están sobre los lados del cuadrado. Si, por ejemplo, el vértice
{\textbackslash(A\textbackslash)} no está en uno de los lados del
cuadrado, entonces podemos prolongar el rayo
{\textbackslash(BA\textbackslash)} para intersectar a un lado del
cuadrado en un punto {\textbackslash(A\textquotesingle\textbackslash)}.
El triángulo {\textbackslash(A\textquotesingle BC\textbackslash)} tiene
mayor área que el {\textbackslash(ABC\textbackslash)} por contenerlo, y
por lo tanto basta con ver que
{\textbackslash(A\textquotesingle BC\textbackslash)} tiene área menor o
igual a {\textbackslash(2\textbackslash)}.

\pandocbounded{\includesvg[keepaspectratio]{../_images/hacia-lado.svg}}

Lo siguiente es argumentar por qué podemos suponer que cada vértice
está sobre un vértice del cuadrado. Supongamos que no. Si
{\textbackslash(A\textbackslash)} está en el lado
{\textbackslash(PQ\textbackslash)} del cuadrado, pero no en el vértice
{\textbackslash(P\textbackslash)} ó {\textbackslash(Q\textbackslash)},
entonces podemos mover a {\textbackslash(A\textbackslash)} hacia
{\textbackslash(P\textbackslash)} o hacia
{\textbackslash(Q\textbackslash)} y preservar el área o aumentarla.

\pandocbounded{\includesvg[keepaspectratio]{../_images/hacia-vertice.svg}}

De este modo, podemos llevar cualquier triángulo a un triángulo
{\textbackslash(ABC\textbackslash)} de área mayor y con vértices sobre
los vértices del cuadrado. Cualquier triángulo así o bien es
degenerado, o bien es de área
{\textbackslash(\textbackslash frac\{2\textbackslash cdot
2\}\{2\}=2\textbackslash)}.
{{\textbackslash(\textbackslash square\textbackslash)}}

Una versión equivalente es la siguiente.

\textbf{Proposición.} Si se tienen {\textbackslash(m\textbackslash)}
objetos o más que se deben repartir en
{\textbackslash(n\textbackslash)} casillas, entonces debe haber al menos
una casilla en donde se colocaron al menos
{\textbackslash(\textbackslash lceil
\textbackslash frac\{m\}\{n\}\textbackslash rceil\textbackslash)}
objetos.

Como ejemplo sencillo, si tenemos {\textbackslash(100\textbackslash)}
personas, entonces hay
{\textbackslash(\textbackslash lceil\textbackslash frac\{100\}\{12\}\textbackslash rceil=9\textbackslash)}
de ellas que tienen el mismo mes de cumpleaños. Veamos un problema un
poco más elaborado.

\textbf{Problema.} Se tienen {\textbackslash(601\textbackslash)}
caballos en un tablero de ajedrez de
{\textbackslash(30\textbackslash times 30\textbackslash)}. Demuestra que
hay dos de ellos que se atacan entre sí.

\emph{Solución.} Tracemos rayas cada 3 filas y cada 3 columnas para
dividir a todo el tablero en {\textbackslash(100\textbackslash)}
regiones de {\textbackslash(3\textbackslash times 3\textbackslash)} cada
una. Como hay {\textbackslash(501\textbackslash)} caballos, entonces hay
una de ellas en donde colocamos por lo menos
{\textbackslash(\textbackslash lceil\textbackslash frac\{501\}\{100\}\textbackslash rceil=6\textbackslash)}
caballos. Veremos que en esta región hay dos caballos que se atacan
entre sí. Esto lo hacemos argumentando mediante otro principio de las
casillas como sigue. Nombremos las casillas del tablero de
{\textbackslash(3\textbackslash times 3\textbackslash)} de la siguiente
manera:

\pandocbounded{\includesvg[keepaspectratio]{../_images/caballos-3.svg}}

Tenemos {\textbackslash(5\textbackslash)} regiones: las de la letra
{\textbackslash(A\textbackslash)}, la letra
{\textbackslash(B\textbackslash)}, la letra
{\textbackslash(C\textbackslash)}, la letra
{\textbackslash(D\textbackslash)} y la letra
{\textbackslash(E\textbackslash)}. Por el principio de las casillas
básico, como hay {\textbackslash(6\textbackslash)} caballos en este
tablero de {\textbackslash(3\textbackslash times 3\textbackslash)}, hay
por lo menos dos en una misma región. No puede ser la región
{\textbackslash(E\textbackslash)} pues esa consiste de una sola casilla.
Con esto terminamos pues cualquiera de las otras regiones consiste de
dos casillas en donde los caballos se pueden atacar entre sí.
{{\textbackslash(\textbackslash square\textbackslash)}}

\phantomsection\label{principio-de-las-casillas-version-infinita}
\subsection{\texorpdfstring{Principio de las casillas, versión
infinita\hyperref[principio-de-las-casillas-version-infinita]{\#}}{Principio de las casillas, versión infinita\#}}\label{principio-de-las-casillas-versiuxe3uxb3n-infinita}

Una última versión del principio de las casillas que veremos ahora es
la versión infinita. Dice lo siguiente.

\textbf{Proposición.} Si se tiene una infinidad de objetos que se deban
repartir en una cantidad finita de casillas, entonces debe haber al
menos una casilla con una infinidad de objetos.

Veamos un ejemplo sencillo. Tomemos
{\textbackslash(\textbackslash pi\textbackslash)} en su expresión
decimal:

\textbackslash{[}\textbackslash pi=3.1415926535\textbackslash ldots\textbackslash{]}

Notemos que el {\textbackslash(5\textbackslash)} aparece tres veces
hasta aquí. ¿Será que aparece una infinidad de veces en la expresión
decimal de {\textbackslash(\textbackslash pi\textbackslash)}? Nadie lo
sabe, es una pregunta abierta. Y el {\textbackslash(5\textbackslash)} no
tiene nada de particular. Para ninguno de los dígitos se sabe si aparece
una infinidad de veces en la expresión decimal de
{\textbackslash(\textbackslash pi\textbackslash)} o no. Sin embargo, sí
hay algo que podemos decir. Como hay una infinidad de dígitos y
únicamente existen los dígitos de {\textbackslash(0\textbackslash)} a
{\textbackslash(9\textbackslash)}, entonces por el principio de las
casillas debe existir uno de ellos que se repite una infinidad de veces
dentro de {\textbackslash(\textbackslash pi\textbackslash)}. ¿Cuál?
Quién sabe.

Terminemos con un problema más que usa el principio de las casillas en
su versión infinita.

\textbf{Problema.} Se tiene un tablero de
{\textbackslash(n\textbackslash)} casillas colocadas horizontalmente. En
cada casilla se ha colocado una flecha, ya sea apuntando hacia la
izquierda, o hacia la derecha. Una hormiga comienza en la casilla de
hasta la izquierda. Se va a mover de acuerdo a las siguientes reglas:

\begin{itemize}
\item
  Si está en la casilla de hasta la izquierda, y la flecha apunta a la
  izquierda, la hormiga no hace nada y la flecha apunta ahora a la
  derecha.
\item
  Si está en la casilla de hasta la derecha, y la flecha apunta a la
  derecha, entonces la hormiga gana el juego y sale del tablero.
\item
  En cualquier otro caso, la hormiga se mueve a la dirección que indica
  la flecha, y ya que pasó esto, la flecha apunta hacia el otro lado.
\end{itemize}

Muestra que sin importar cómo están colocadas las flechas
inicialmente, la hormiga gana.

Quizás quieras jugar un poco con el problema antes de ver la solución.
La siguiente función recibe como argumentos la variable {\texttt{n}}
que da el tamaño del tablero, la variable {\texttt{flechas}} a la que
se le puede poner la configuración inicial que se desee y la variable
{\texttt{pasos}} que indica cuántos pasos se quieren dar como máximo.
Va mostrando el tablero, en donde la {\texttt{x}} indica a la hormiga y
avisa cuando haya salido.

\begin{verbatim}
def hormiga(n,tablero,maxpasos):
    posicion=0
    tablero=list(tablero)
    for j in range(maxpasos):
        if posicion==0 and tablero[posicion]=="<":
            tablero[posicion]=">"
        elif posicion==n-1 and tablero[posicion]==">":
            print("La hormiga salió")
            return
        else:
            if tablero[posicion]==">":
                tablero[posicion]="<"
                posicion+=1
            elif tablero[posicion]=="<":
                tablero[posicion]=">"
                posicion-=1
        print(posicion*' '+'x'+(n-posicion-1)*" "+'\n'+"".join(tablero))

hormiga(4,'<<>>',10)
\end{verbatim}

\begin{verbatim}
x   
><>>
 x  
<<>>
x   
<>>>
x   
>>>>
 x  
<>>>
  x 
<<>>
   x
<<<>
La hormiga salió
\end{verbatim}

\emph{Solución.} Vamos a demostrar que la hormiga siempre sale. Para
ello, procederemos por contradicción. Supongamos que la hormiga nunca
sale. Entonces, se queda siempre dentro del tablero, dando una infinidad
de pasos. Como sólo hay una cantidad finita
{\textbackslash(n\textbackslash)} de casillas dentro del tablero,
entonces hay alguna casilla {\textbackslash(C\textbackslash)} que la
hormiga visita una infinidad de veces.

Pero como en cada visita las flechas de
{\textbackslash(C\textbackslash)} alternan dirección, entonces también
se visita una infinidad de veces las casillas adyacentes a
{\textbackslash(C\textbackslash)}. Pero este argumento aplica también
para las adyacentes a ellas, y en general, para todas las casillas. De
este modo, todas las casillas son visitadas una infinidad de veces. En
partícular, la casilla de hasta la derecha es visitada una infinidad de
veces. Así, como mucho tras dos veces de visitarla la hormiga saldría
del tablero. Esto da la contradicción buscada.
{{\textbackslash(\textbackslash square\textbackslash)}}

\subsection{\texorpdfstring{Tarea
moral\hyperref[tarea-moral]{\#}}{Tarea moral\#}}\label{tarea-moral}

Los siguientes problemas te ayudarán a practicar lo visto en esta
entrada. Para resolverlos, necesitarás usar herramientas matemáticas,
computacionales o ambas.

\begin{enumerate}
\item
  Investiga aproximadamente cuántos pelos tiene una persona en el
  cabello. Usa esto y el principio de las casillas para mostrar que en
  México hay dos personas con el mismo número de pelos en el cabello.
  ¿Habrá {\textbackslash(100\textbackslash)} personas con la misma
  cantidad de pelos en el cabello? Explica tus suposiciones.
\item
  ¿Cuál es la máxima cantidad de alfiles que se pueden colocar en un
  tablero de ajedrez sin que se ataquen entre sí?
\item
  Demuestra que si se tienen {\textbackslash(9\textbackslash)} puntos
  con coordenadas enteras en
  {\textbackslash(\textbackslash mathbb\{R\}\^{}3\textbackslash)},
  entonces hay dos de ellos cuyo punto medio también tiene coordenadas
  enteras.
\item
  Demuestra que para cualquier entero positivo
  {\textbackslash(k\textbackslash)} se tiene que la sucesión de
  Fibonacci tiene una infinidad de múltiplos de
  {\textbackslash(k\textbackslash)}.
\item
  Sea {\textbackslash(k\textbackslash)} un entero positivo. Se pintan
  todos los puntos con coordenadas enteras del plano con
  {\textbackslash(k\textbackslash)} colores. Muestra que sin importar
  cómo sea la coloración, siempre existe un rectángulo con lados
  paralelos a los ejes y cuyos vértices son cuatro puntos de
  coordenadas enteras pintados del mismo color.
\end{enumerate}

\href{BuscarPatron.html}{\emph{}}

anterior

Buscar un patrón

\href{DobleConteo.html}{}

siguiente

Principio de doble conteo y coeficientes binomiales

\emph{}

\emph{} Contenido

\begin{itemize}
\tightlist
\item
  \hyperref[introduccion]{Introducción}
\item
  \hyperref[principio-de-las-casillas-version-basica]{Principio de las
  casillas, versión básica}
\item
  \hyperref[principio-de-las-casillas-version-con-mas-objetos]{Principio
  de las casillas, versión con más objetos}
\item
  \hyperref[principio-de-las-casillas-version-infinita]{Principio de las
  casillas, versión infinita}
\item
  \hyperref[tarea-moral]{Tarea moral}
\end{itemize}

Por Leonardo Ignacio Martínez Sandoval

© Copyright 2022.\\

\end{document}
